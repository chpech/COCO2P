% generated by GAPDoc2LaTeX from XML source (Frank Luebeck)
\documentclass[a4paper,11pt]{report}

\usepackage[top=37mm,bottom=37mm,left=27mm,right=27mm]{geometry}
\sloppy
\pagestyle{myheadings}
\usepackage{amssymb}
\usepackage[latin1]{inputenc}
\usepackage{makeidx}
\makeindex
\usepackage{color}
\definecolor{FireBrick}{rgb}{0.5812,0.0074,0.0083}
\definecolor{RoyalBlue}{rgb}{0.0236,0.0894,0.6179}
\definecolor{RoyalGreen}{rgb}{0.0236,0.6179,0.0894}
\definecolor{RoyalRed}{rgb}{0.6179,0.0236,0.0894}
\definecolor{LightBlue}{rgb}{0.8544,0.9511,1.0000}
\definecolor{Black}{rgb}{0.0,0.0,0.0}

\definecolor{linkColor}{rgb}{0.0,0.0,0.554}
\definecolor{citeColor}{rgb}{0.0,0.0,0.554}
\definecolor{fileColor}{rgb}{0.0,0.0,0.554}
\definecolor{urlColor}{rgb}{0.0,0.0,0.554}
\definecolor{promptColor}{rgb}{0.0,0.0,0.589}
\definecolor{brkpromptColor}{rgb}{0.589,0.0,0.0}
\definecolor{gapinputColor}{rgb}{0.589,0.0,0.0}
\definecolor{gapoutputColor}{rgb}{0.0,0.0,0.0}

%%  for a long time these were red and blue by default,
%%  now black, but keep variables to overwrite
\definecolor{FuncColor}{rgb}{0.0,0.0,0.0}
%% strange name because of pdflatex bug:
\definecolor{Chapter }{rgb}{0.0,0.0,0.0}
\definecolor{DarkOlive}{rgb}{0.1047,0.2412,0.0064}


\usepackage{fancyvrb}

\usepackage{mathptmx,helvet}
\usepackage[T1]{fontenc}
\usepackage{textcomp}


\usepackage[
            pdftex=true,
            bookmarks=true,        
            a4paper=true,
            pdftitle={Written with GAPDoc},
            pdfcreator={LaTeX with hyperref package / GAPDoc},
            colorlinks=true,
            backref=page,
            breaklinks=true,
            linkcolor=linkColor,
            citecolor=citeColor,
            filecolor=fileColor,
            urlcolor=urlColor,
            pdfpagemode={UseNone}, 
           ]{hyperref}

\newcommand{\maintitlesize}{\fontsize{50}{55}\selectfont}

% write page numbers to a .pnr log file for online help
\newwrite\pagenrlog
\immediate\openout\pagenrlog =\jobname.pnr
\immediate\write\pagenrlog{PAGENRS := [}
\newcommand{\logpage}[1]{\protect\write\pagenrlog{#1, \thepage,}}
%% were never documented, give conflicts with some additional packages

\newcommand{\GAP}{\textsf{GAP}}

%% nicer description environments, allows long labels
\usepackage{enumitem}
\setdescription{style=nextline}

%% depth of toc
\setcounter{tocdepth}{1}





%% command for ColorPrompt style examples
\newcommand{\gapprompt}[1]{\color{promptColor}{\bfseries #1}}
\newcommand{\gapbrkprompt}[1]{\color{brkpromptColor}{\bfseries #1}}
\newcommand{\gapinput}[1]{\color{gapinputColor}{#1}}


\begin{document}

\logpage{[ 0, 0, 0 ]}
\begin{titlepage}
\mbox{}\vfill

\begin{center}{\maintitlesize \textbf{The \textsf{COCO2P} Package\mbox{}}}\\
\vfill

\hypersetup{pdftitle=The \textsf{COCO2P} Package}
\markright{\scriptsize \mbox{}\hfill The \textsf{COCO2P} Package \hfill\mbox{}}
{\Huge Version 0.17\mbox{}}\\[1cm]
\mbox{}\\[2cm]
{\Large \textbf{Mikhail Klin  \mbox{}}}\\
{\Large \textbf{Christian Pech  \mbox{}}}\\
{\Large \textbf{Sven Reichard  \mbox{}}}\\
\hypersetup{pdfauthor=Mikhail Klin  ; Christian Pech  ; Sven Reichard  }
\end{center}\vfill

\mbox{}\\
{\mbox{}\\
\small \noindent \textbf{Mikhail Klin  }  Email: \href{mailto://klin@math.bgu.ac.il } {\texttt{klin@math.bgu.ac.il }}}\\
{\mbox{}\\
\small \noindent \textbf{Christian Pech  }  Email: \href{mailto://christian.pech@tu-dresden.de} {\texttt{christian.pech@tu-dresden.de}}}\\
{\mbox{}\\
\small \noindent \textbf{Sven Reichard  }  Email: \href{mailto://sven.reichard@tu-dresden.de} {\texttt{sven.reichard@tu-dresden.de}}}\\
\end{titlepage}

\newpage\setcounter{page}{2}
{\small 
\section*{Copyright}
\logpage{[ 0, 0, 1 ]}
 {\copyright} 2012, 2014, 2018 by the authors

 This package may be distributed under the terms and conditions of the GNU
Public License Version 2 or higher. \mbox{}}\\[1cm]
\newpage

\def\contentsname{Contents\logpage{[ 0, 0, 2 ]}}

\tableofcontents
\newpage

 
\chapter{\textcolor{Chapter }{Color Graphs}}\logpage{[ 1, 0, 0 ]}
\hyperdef{L}{X781EAC478657D5C5}{}
{
  
\section{\textcolor{Chapter }{Theory}}\label{sec:theory}
\logpage{[ 1, 1, 0 ]}
\hyperdef{L}{X8729B87B848E3F89}{}
{
  A color graph (cgr) in \textsf{COCO2P} is a triple $(V,C,f)$, where $V$ is a set of vertices, $C$ is set of colors, and $f : V \times V \to C$ assigns to every arc its color. \textsf{COCO2P} does not know the concept of non-arcs. However, this is not an essential
restriction, since non-arcs may be simulated by introducing a special
distinguished color. 

 Of special interrest in \textsf{COCO2P} are WL-stable color graphs (that is cgrs that are stable under the
Weisfeiler-Leman algorithm \cite{WeiL68}, \cite{Wei76}). In the frames of \textsf{COCO2P} the maximal monochromatic sets of arcs of a WL-stable color graph will always
form a \emph{coherent configuration}. On the other hand, from every coherent configuration we can obtain a
WL-stable cgr (every pair of vertices is colored by the relation it belongs
to). 

 For historical reasons, \textsf{COCO2P} uses the nomenclature of color graphs. However, this is only of importance for
concepts like automorphisms and isomorphisms. While both, automorphisms and
isomorphisms have to preserve colors (as it is expected for color graphs),
color-automorphisms, and color-isomorphisms only have to respect color
classes, that is, they may map arcs of one color to arcs of another color. 

 }

 
\section{\textcolor{Chapter }{ On the representation of color graphs in \textsf{COCO2P} }}\label{sec:cgr-representation}
\logpage{[ 1, 2, 0 ]}
\hyperdef{L}{X8376827B806E61A4}{}
{
  For a color graph, the set of vertices as well as the set of colors may be any
finite set representable in \textsf{GAP}. For performance reasons, \textsf{COCO2P} does not use these sets inside its algorithms (except when constructing color
graphs). Instead, \textsf{COCO2P} refers to vertices and colors by their position in the vertex-set and
color-set, respectively. In fact vertices and colors are identified with these
indices. In order not to loose information, every color graph in \textsf{COCO2P} keeps a list of names of vertices and a list of names of colors. The set of
vertex-names is equal to the original set of vertices, and the set of color
names is equal to the original set of colors. }

 
\section{\textcolor{Chapter }{Functions for the construction of color graphs}}\label{sec:cgr-constructors}
\logpage{[ 1, 3, 0 ]}
\hyperdef{L}{X7B42CC2287F1C3FB}{}
{
  

\subsection{\textcolor{Chapter }{ColorGraphByOrbitals}}
\logpage{[ 1, 3, 1 ]}\nobreak
\hyperdef{L}{X7DCB8AAF7946F031}{}
{\noindent\textcolor{FuncColor}{$\triangleright$\enspace\texttt{ColorGraphByOrbitals({\mdseries\slshape grp[, domain[, Action[, completeDom]]]})\index{ColorGraphByOrbitals@\texttt{ColorGraphByOrbitals}}
\label{ColorGraphByOrbitals}
}\hfill{\scriptsize (function)}}\\


 This function constructs the color graph of orbitals color graphs from a group
action by mapping each arc to a representative of its orbital of the given
group action. 

 In its first form, the function returns the color graph of orbitals of the
permutation group \mbox{\texttt{\mdseries\slshape grp}} in its natural action (i.e. on $\{1,\dots,n\}$, where $n$ is the largest moved point of \mbox{\texttt{\mdseries\slshape grp}}). 
\begin{Verbatim}[commandchars=!@|,fontsize=\small,frame=single,label=Example]
  !gapprompt@gap>| !gapinput@d7 := Group( (1,2,3,4,5,6,7), (1,7)(2,6)(3,5));;|
  !gapprompt@gap>| !gapinput@cgr := ColorGraphByOrbitals(d7);|
  <color graph of order 7 and rank 4>
  	    
\end{Verbatim}
 

 In the second form the function returns the color graph of orbitals of \mbox{\texttt{\mdseries\slshape grp}} acting on \mbox{\texttt{\mdseries\slshape domain}} \texttt{OnPoints}. If \mbox{\texttt{\mdseries\slshape domain}} is not invariant under \mbox{\texttt{\mdseries\slshape grp}}, then the smallest invariant extension of \mbox{\texttt{\mdseries\slshape domain}} is taken as acting domain. 

 
\begin{Verbatim}[commandchars=!@|,fontsize=\small,frame=single,label=Example]
  !gapprompt@gap>| !gapinput@d7 := Group( (1,2,3,4,5,6,7), (1,7)(2,6)(3,5));;|
  !gapprompt@gap>| !gapinput@cgr := ColorGraph(d7, [1]);|
  <color graph of order 7 and rank 4>
  	    
\end{Verbatim}
 

 In the third variant of \texttt{ColorGraphByOrbitals} an action can be given: 
\begin{Verbatim}[commandchars=!@|,fontsize=\small,frame=single,label=Example]
  !gapprompt@gap>| !gapinput@cgr:=ColorGraph(SymmetricGroup(5), Combinations([1..5],2), OnSets);|
  <color graph of order 10 and rank 3>
  	    
\end{Verbatim}
 

 The optional fourth argument \mbox{\texttt{\mdseries\slshape completeDom}} is a boolean. If it is \texttt{True}, then the function assumes that \mbox{\texttt{\mdseries\slshape domain}} is closed under \mbox{\texttt{\mdseries\slshape action}} of \mbox{\texttt{\mdseries\slshape grp}}. This has the effect, that the function dows not try to complete it. The
effect is that in the resulting color graph it is guaranteed that the vertex
with number \texttt{i} corresponds exactly to \texttt{domain[i]}. }

 

\subsection{\textcolor{Chapter }{ColorGraph}}
\logpage{[ 1, 3, 2 ]}\nobreak
\hyperdef{L}{X7DF2D0B07C280D55}{}
{\noindent\textcolor{FuncColor}{$\triangleright$\enspace\texttt{ColorGraph({\mdseries\slshape grp[, domain[, action[, completeDom[, coloring]]]]})\index{ColorGraph@\texttt{ColorGraph}}
\label{ColorGraph}
}\hfill{\scriptsize (function)}}\\


 This is the most general function for the construction of color graphs. When
called with less than $5$ arguments, it is identical with the function \texttt{ColorGraphByOrbitals} (\ref{ColorGraphByOrbitals}) 

 The optional fifth argument \mbox{\texttt{\mdseries\slshape coloring}} is a coloring-function. It takes as input two vertices (elements of the acting
domain) $u, v$, and it has to return the color of the arc $(u, v)$. In principle, the color can be any \textsf{GAP} object. However, it should be possible to compare colors and to form sets of
them. 
\begin{Verbatim}[commandchars=!@|,fontsize=\small,frame=single,label=Example]
  !gapprompt@gap>| !gapinput@cgr:=ColorGraph(SymmetricGroup(8),|
  !gapprompt@>| !gapinput@Combinations([1..8],4), OnSets, true,|
  !gapprompt@>| !gapinput@function(u,v) return|
  !gapprompt@>| !gapinput@Length(Intersection(u,v));end);|
  <color graph of order 70 and rank 5>
  	    
\end{Verbatim}
 It is supposed that \mbox{\texttt{\mdseries\slshape coloring}} is invariant under the given action (this is not checked!). }

 

\subsection{\textcolor{Chapter }{ColorGraphByMatrix}}
\logpage{[ 1, 3, 3 ]}\nobreak
\hyperdef{L}{X858BF76A7DC9D2E1}{}
{\noindent\textcolor{FuncColor}{$\triangleright$\enspace\texttt{ColorGraphByMatrix({\mdseries\slshape mat})\index{ColorGraphByMatrix@\texttt{ColorGraphByMatrix}}
\label{ColorGraphByMatrix}
}\hfill{\scriptsize (function)}}\\


 This function constructs a color graph from its adjacency matrix. The argument \mbox{\texttt{\mdseries\slshape mat}} is a list of $n$ lists of length $n$. The vertex-set of the resulting color graph is $\{1,\dots,n\}$, while the color of the arc $(i,j)$ is \texttt{mat[i][j]}. The entries can be any kind of \textsf{GAP}-objects that can be compared and that can be organized in a set. 
\begin{Verbatim}[commandchars=!@|,fontsize=\small,frame=single,label=Example]
  !gapprompt@gap>| !gapinput@m:=[["black","red"  ,"blue" ,"blue" ,"blue" ],|
  !gapprompt@>| !gapinput@       ["blue" ,"black","red"  ,"blue" ,"blue" ],|
  !gapprompt@>| !gapinput@       ["blue" ,"blue", "black","red"  ,"blue" ],|
  !gapprompt@>| !gapinput@       ["blue" ,"blue", "blue" ,"black","red"  ],|
  !gapprompt@>| !gapinput@       ["red"  ,"blue", "blue" ,"blue" ,"black"]];;|
  !gapprompt@gap>| !gapinput@cgr:=ColorGraphByMatrix(m);|
  <color graph of order 5 and rank 3>
  	    
\end{Verbatim}
 }

 

\subsection{\textcolor{Chapter }{ColorGraphByWLStabilization}}
\logpage{[ 1, 3, 4 ]}\nobreak
\hyperdef{L}{X85EAC9AD7F1758E6}{}
{\noindent\textcolor{FuncColor}{$\triangleright$\enspace\texttt{ColorGraphByWLStabilization({\mdseries\slshape cgr})\index{ColorGraphByWLStabilization@\texttt{ColorGraphByWLStabilization}}
\label{ColorGraphByWLStabilization}
}\hfill{\scriptsize (function)}}\\


 If \mbox{\texttt{\mdseries\slshape cgr}} is WL-stable then the function returns \mbox{\texttt{\mdseries\slshape cgr}}. Otherwise, the WL-stabilization of \mbox{\texttt{\mdseries\slshape cgr}} is returned. The colors of the stabilization have names of the shape \texttt{[c,i]} where \texttt{c} is a color of \mbox{\texttt{\mdseries\slshape cgr}} and \texttt{i} is the index of a fragment of color \texttt{c}. 

 This function does not really implement the Weisfeiler-Leman algorithm. Rather
it does a stabilization inside of a Schurian WL-stable fission of \mbox{\texttt{\mdseries\slshape cgr}}. The performance depends mainly on the order of the group of known
automorphisms of \mbox{\texttt{\mdseries\slshape cgr}} (cf \texttt{KnownGroupOfAutomorphisms} (\ref{KnownGroupOfAutomorphisms:for color graphs})). }

 

\subsection{\textcolor{Chapter }{WLStableColorGraphByMatrix}}
\logpage{[ 1, 3, 5 ]}\nobreak
\hyperdef{L}{X7FDD766E79BFEAF0}{}
{\noindent\textcolor{FuncColor}{$\triangleright$\enspace\texttt{WLStableColorGraphByMatrix({\mdseries\slshape mat})\index{WLStableColorGraphByMatrix@\texttt{WLStableColorGraphByMatrix}}
\label{WLStableColorGraphByMatrix}
}\hfill{\scriptsize (function)}}\\


 This function gets as input a square matrix \mbox{\texttt{\mdseries\slshape mat}} and returns the color graph of the Weisfeiler-Leman- stabilization of \mbox{\texttt{\mdseries\slshape Mat}}. The entries of \mbox{\texttt{\mdseries\slshape mat}} can be any kind of \textsf{GAP}-objects that can be compared and that can be organized in a set. The
vertex-set of the resulting color graph is $\{1,\dots,n\}$, while the color of the arc $(i,j)$ is \texttt{[mat[i][j],k]}, where \texttt{k} is a positive integer. 

 This constructor works usually much faster then the combination of \texttt{ColorGraphByMatrix} (\ref{ColorGraphByMatrix}) and \texttt{ColorGraphByWLStabilization} (\ref{ColorGraphByWLStabilization}). 
\begin{Verbatim}[commandchars=!@|,fontsize=\small,frame=single,label=Example]
  !gapprompt@gap>| !gapinput@c:=AllAssociationSchemes(10)[3];|
  AS(10,3)
  !gapprompt@gap>| !gapinput@a:=AdjacencyMatrix(c);;|
  !gapprompt@gap>| !gapinput@Display(a);|
  [ [  1,  2,  2,  2,  3,  3,  3,  3,  3,  3 ],
    [  2,  1,  3,  3,  2,  2,  3,  3,  3,  3 ],
    [  2,  3,  1,  3,  3,  3,  2,  2,  3,  3 ],
    [  2,  3,  3,  1,  3,  3,  3,  3,  2,  2 ],
    [  3,  2,  3,  3,  1,  3,  2,  3,  2,  3 ],
    [  3,  2,  3,  3,  3,  1,  3,  2,  3,  2 ],
    [  3,  3,  2,  3,  2,  3,  1,  3,  3,  2 ],
    [  3,  3,  2,  3,  3,  2,  3,  1,  2,  3 ],
    [  3,  3,  3,  2,  2,  3,  3,  2,  1,  3 ],
    [  3,  3,  3,  2,  3,  2,  2,  3,  3,  1 ] ]
  !gapprompt@gap>| !gapinput@a[1][1]:=4;;|
  !gapprompt@gap>| !gapinput@c1:=ColorGraphByMatrix(a);|
  <color graph of order 10 and rank 4>
  !gapprompt@gap>| !gapinput@c2:=WLStableColorGraphByMatrix(a);|
  <color graph of order 10 and rank 15>
  !gapprompt@gap>| !gapinput@Display(c1);|
  [ [  4,  2,  2,  2,  3,  3,  3,  3,  3,  3 ],
    [  2,  1,  3,  3,  2,  2,  3,  3,  3,  3 ],
    [  2,  3,  1,  3,  3,  3,  2,  2,  3,  3 ],
    [  2,  3,  3,  1,  3,  3,  3,  3,  2,  2 ],
    [  3,  2,  3,  3,  1,  3,  2,  3,  2,  3 ],
    [  3,  2,  3,  3,  3,  1,  3,  2,  3,  2 ],
    [  3,  3,  2,  3,  2,  3,  1,  3,  3,  2 ],
    [  3,  3,  2,  3,  3,  2,  3,  1,  2,  3 ],
    [  3,  3,  3,  2,  2,  3,  3,  2,  1,  3 ],
    [  3,  3,  3,  2,  3,  2,  2,  3,  3,  1 ] ]
  !gapprompt@gap>| !gapinput@Display(c2);|
  [[[4,1],[2,1],[2,1],[2,1],[3,1],[3,1],[3,1],[3,1],[3,1],[3,1]],
   [[2,2],[1,2],[3,4],[3,4],[2,5],[2,5],[3,6],[3,6],[3,6],[3,6]],
   [[2,2],[3,4],[1,2],[3,4],[3,6],[3,6],[2,5],[2,5],[3,6],[3,6]],
   [[2,2],[3,4],[3,4],[1,2],[3,6],[3,6],[3,6],[3,6],[2,5],[2,5]],
   [[3,2],[2,4],[3,5],[3,5],[1,1],[3,3],[2,3],[3,7],[2,3],[3,7]],
   [[3,2],[2,4],[3,5],[3,5],[3,3],[1,1],[3,7],[2,3],[3,7],[2,3]],
   [[3,2],[3,5],[2,4],[3,5],[2,3],[3,7],[1,1],[3,3],[3,7],[2,3]],
   [[3,2],[3,5],[2,4],[3,5],[3,7],[2,3],[3,3],[1,1],[2,3],[3,7]],
   [[3,2],[3,5],[3,5],[2,4],[2,3],[3,7],[3,7],[2,3],[1,1],[3,3]],
   [[3,2],[3,5],[3,5],[2,4],[3,7],[2,3],[2,3],[3,7],[3,3],[1,1]]]
  	    
\end{Verbatim}
 }

 

\subsection{\textcolor{Chapter }{ClassicalCompleteAffineScheme}}
\logpage{[ 1, 3, 6 ]}\nobreak
\hyperdef{L}{X78E3485F85B414A5}{}
{\noindent\textcolor{FuncColor}{$\triangleright$\enspace\texttt{ClassicalCompleteAffineScheme({\mdseries\slshape q})\index{ClassicalCompleteAffineScheme@\texttt{ClassicalCompleteAffineScheme}}
\label{ClassicalCompleteAffineScheme}
}\hfill{\scriptsize (function)}}\\


 The classical complete affine scheme is a WL-stable, Schurian, amorphic color
graph defined on the set of points of the affine plane over $GF(q)$. The reflexive closure of every irreflexive color class is an equivalence
relation whose equivalence classes form a complete parallel class of lines.
Moreover, to every parallel class there corresponds a color class. 

 This function returns the classical complete affine scheme over $GF(q)$. }

 

\subsection{\textcolor{Chapter }{JohnsonScheme}}
\logpage{[ 1, 3, 7 ]}\nobreak
\hyperdef{L}{X7D6476BA7C79C9DA}{}
{\noindent\textcolor{FuncColor}{$\triangleright$\enspace\texttt{JohnsonScheme({\mdseries\slshape n, k})\index{JohnsonScheme@\texttt{JohnsonScheme}}
\label{JohnsonScheme}
}\hfill{\scriptsize (function)}}\\


 The Johnson scheme $J(n,k)$ is a WL-stable, Schurian color graph. Its vertices are the \mbox{\texttt{\mdseries\slshape k}}-element subsets of $\{1,\dots,n\}$. The colors are elements of $\{0,\dots,k\}$. The color of an arc $(M,N)$ is the cardinality of the intersection of $M$ and $N$. 

 This function returns the Johnson scheme $J(n,k)$. }

 

\subsection{\textcolor{Chapter }{CyclotomicColorGraph}}
\logpage{[ 1, 3, 8 ]}\nobreak
\hyperdef{L}{X7F9D294A789CBE87}{}
{\noindent\textcolor{FuncColor}{$\triangleright$\enspace\texttt{CyclotomicColorGraph({\mdseries\slshape p, n, d})\index{CyclotomicColorGraph@\texttt{CyclotomicColorGraph}}
\label{CyclotomicColorGraph}
}\hfill{\scriptsize (function)}}\\


 Let $p$ be a prime, $n$, $d$ be positive integers, such that $d$ divides $(p^n-1)$. Let $q:=p^n$, and let $r$ be a primitive element of $GF(q)$. Let $C$ be the set of all powers of $r^d$ in $GF(q)$ the cyclotomic colored graph $Cyc(p,n,d)$ has as vertices the elements of $GF(q)$. The set of colors is given by $\{*,0,1,...,d-1\}$. A pair $(x,y)$ of vertices has color $*$ in $Cyc(p,n,d)$ if $x=y$. It has color $i$ if $(x-y)$ is an element of $C\cdot(r^i)$. 

 This function returns the Cyclotomic scheme $Cyc(p,n,d)$. }

 

\subsection{\textcolor{Chapter }{BIKColorGraph}}
\logpage{[ 1, 3, 9 ]}\nobreak
\hyperdef{L}{X7EAC537482E4A685}{}
{\noindent\textcolor{FuncColor}{$\triangleright$\enspace\texttt{BIKColorGraph({\mdseries\slshape m})\index{BIKColorGraph@\texttt{BIKColorGraph}}
\label{BIKColorGraph}
}\hfill{\scriptsize (function)}}\\


 This function generates the color graphs described in the paper \cite{BroIvaKli89}. These color graphs are interesting because they may be used to construct $3$-isoregular strongly regular graphs with the $5$-vertex condition. The vertex set of \texttt{BIKColorGraph(m)} is $V=GF(2)^{2m}$. For the description of colors of the arcs consider a quadratic form $q$ of Witt-index $m$ on $V$. Let $Q$ be the quadric defined by $q$, and let $S$ be a maximal singular subspace of $q$. A pair of vectors $(v,w)$ is colored by 
\begin{description}
\item[{\texttt{"=":}}] if $v = w$,
\item[{\texttt{"Q+S+":}}]  if $v + w \in S$,
\item[{\texttt{"Q+S-":}}] if $v + w \in Q \setminus S$,
\item[{\texttt{"Q-":}}] if $v + w \notin Q$.
\end{description}
 The following code constructs the Ivanov-graph on $256$ vertices. This was historically the first strongly regular graph to be found
that is non-rank-$3$ and that satisfies the $5$-vertex condition (cf. \cite{Iva89}). 
\begin{Verbatim}[commandchars=!@|,fontsize=\small,frame=single,label=Example]
  !gapprompt@gap>| !gapinput@cgr:=BIKColorGraph(4);|
  <color graph of order 256 and rank 4>
  !gapprompt@gap>| !gapinput@ColorNames(cgr);|
  [ "=", "Q+S+", "Q+S-", "Q-" ]
  !gapprompt@gap>| !gapinput@gamma:=BaseGraphOfColorGraph(cgr,3);;|
  !gapprompt@gap>| !gapinput@IsStronglyRegular(gamma);|
  true
  !gapprompt@gap>| !gapinput@gamma.srg;|
  rec( k := 120, lambda := 56, mu := 56, r := 8, s := -8, v := 256 )
          
\end{Verbatim}
 }

 

\subsection{\textcolor{Chapter }{IvanovColorGraph}}
\logpage{[ 1, 3, 10 ]}\nobreak
\hyperdef{L}{X7C500BC282B17242}{}
{\noindent\textcolor{FuncColor}{$\triangleright$\enspace\texttt{IvanovColorGraph({\mdseries\slshape m})\index{IvanovColorGraph@\texttt{IvanovColorGraph}}
\label{IvanovColorGraph}
}\hfill{\scriptsize (function)}}\\


 This function generates a series of color graphs described in \cite{Iva94}. These color graphs may be used to construct $3$-isoregular strongly regular graphs with the $5$-vertex condition. These color graphs are interesting because they may be used
to construct $3$-isoregular strongly regular graphs with the $5$-vertex condition. The vertex set of \texttt{IvanovColorGraph(m)} is $V=GF(2)^{2m}$. For the description of colors of the arcs consider a quadratic form $q$ of Witt-index $m-1$ on $V$. Let $Q$ be the quadric defined by $q$, let $S$ be a maximal singular subspace of $q$, and let $O$ be the orthogonal complement of $S$. A pair of vectors $(v,w)$ is colored by 
\begin{description}
\item[{\texttt{"=":}}] if $v = w$,
\item[{\texttt{"Q+S+":}}]  if $v + w \in S$,
\item[{\texttt{"Q+S-":}}] if $v + w \in Q \setminus S$,
\item[{\texttt{"Q-O+":}}] if $v + w \in O$.
\item[{\texttt{"Q-O-":}}] if $v + w \notin O \cup Q$.
\end{description}
 
\begin{Verbatim}[commandchars=!@|,fontsize=\small,frame=single,label=Example]
  !gapprompt@gap>| !gapinput@cgr:=IvanovColorGraph(5);|
  <color graph of order 1024 and rank 5>
  !gapprompt@gap>| !gapinput@ColorNames(cgr);|
  [ "=", "Q+S+", "Q+S-", "Q-O+", "Q-O-" ]
  !gapprompt@gap>| !gapinput@gamma:=BaseGraphOfColorGraph(cgr,[2,5]);;|
  !gapprompt@gap>| !gapinput@IsStronglyRegular(gamma);;|
  !gapprompt@gap>| !gapinput@gamma.srg;|
  rec( k := 495, lambda := 238, mu := 240, r := 15, s := -17, v := 1024 )
            
\end{Verbatim}
 }

 

\subsection{\textcolor{Chapter }{AllAssociationSchemes}}
\logpage{[ 1, 3, 11 ]}\nobreak
\hyperdef{L}{X8779C97084BB92B6}{}
{\noindent\textcolor{FuncColor}{$\triangleright$\enspace\texttt{AllAssociationSchemes({\mdseries\slshape n})\index{AllAssociationSchemes@\texttt{AllAssociationSchemes}}
\label{AllAssociationSchemes}
}\hfill{\scriptsize (function)}}\\


 This function creates an interface to the database of small association
schemes by Akihide Hanaki and Izumi Miyamoto from \href{http://math.shinshu-u.ac.jp/~hanaki/as/} {\texttt{http://math.shinshu-u.ac.jp/\texttt{\symbol{126}}hanaki/as/}} (further refered to as the Japanese catalogue) 

 This function downloads the list of small association schemes of order \mbox{\texttt{\mdseries\slshape n}}. Then it converts them to the internal format of \textsf{COCO2P} and returns the resulting list. Every color graph has a name of the shape \texttt{AS(n,k)} where \texttt{k} is the index of the scheme in the list of schemes of order \mbox{\texttt{\mdseries\slshape n}} in the Japanese catalogue. }

 

\subsection{\textcolor{Chapter }{AllCoherentConfigurations}}
\logpage{[ 1, 3, 12 ]}\nobreak
\hyperdef{L}{X867455368202CEE9}{}
{\noindent\textcolor{FuncColor}{$\triangleright$\enspace\texttt{AllCoherentConfigurations({\mdseries\slshape n})\index{AllCoherentConfigurations@\texttt{AllCoherentConfigurations}}
\label{AllCoherentConfigurations}
}\hfill{\scriptsize (function)}}\\


 This function creates an interface to the database of small coherent
configurations on at most 15 vertices by Matan Ziv-Av This function downloads
the list of small coherent configurations of order \mbox{\texttt{\mdseries\slshape n}}. Then it converts them to the internal format of \textsf{COCO2P} and returns the resulting list. Every color graph has a name of the shape \texttt{CC(n,k)} where \texttt{k} is the index of the scheme in the list of schemes of order \mbox{\texttt{\mdseries\slshape n}} in Matan's catalogue. }

 }

 
\section{\textcolor{Chapter }{Functions for the inspection of color graphs}}\label{sec:cgr-inspection}
\logpage{[ 1, 4, 0 ]}
\hyperdef{L}{X79B04A5F80234A2D}{}
{
  

\subsection{\textcolor{Chapter }{OrderOfColorGraph}}
\logpage{[ 1, 4, 1 ]}\nobreak
\hyperdef{L}{X807C46F481D7086B}{}
{\noindent\textcolor{FuncColor}{$\triangleright$\enspace\texttt{OrderOfColorGraph({\mdseries\slshape cgr})\index{OrderOfColorGraph@\texttt{OrderOfColorGraph}}
\label{OrderOfColorGraph}
}\hfill{\scriptsize (attribute)}}\\
\noindent\textcolor{FuncColor}{$\triangleright$\enspace\texttt{OrderOfCocoObject({\mdseries\slshape cgr})\index{OrderOfCocoObject@\texttt{OrderOfCocoObject}!for color graphs}
\label{OrderOfCocoObject:for color graphs}
}\hfill{\scriptsize (attribute)}}\\
\noindent\textcolor{FuncColor}{$\triangleright$\enspace\texttt{Order({\mdseries\slshape cgr})\index{Order@\texttt{Order}!for color graphs}
\label{Order:for color graphs}
}\hfill{\scriptsize (attribute)}}\\


 Returns the number of vertices of \mbox{\texttt{\mdseries\slshape cgr}}. }

 

\subsection{\textcolor{Chapter }{RankOfColorGraph}}
\logpage{[ 1, 4, 2 ]}\nobreak
\hyperdef{L}{X843D2FA779BA5DCC}{}
{\noindent\textcolor{FuncColor}{$\triangleright$\enspace\texttt{RankOfColorGraph({\mdseries\slshape cgr})\index{RankOfColorGraph@\texttt{RankOfColorGraph}}
\label{RankOfColorGraph}
}\hfill{\scriptsize (attribute)}}\\
\noindent\textcolor{FuncColor}{$\triangleright$\enspace\texttt{Rank({\mdseries\slshape cgr})\index{Rank@\texttt{Rank}!for color graphs}
\label{Rank:for color graphs}
}\hfill{\scriptsize (method)}}\\


 Returns the number of colors of \mbox{\texttt{\mdseries\slshape cgr}}. }

 

\subsection{\textcolor{Chapter }{VertexNamesOfCocoObject (for color graphs)}}
\logpage{[ 1, 4, 3 ]}\nobreak
\hyperdef{L}{X83F7E42D834CCF92}{}
{\noindent\textcolor{FuncColor}{$\triangleright$\enspace\texttt{VertexNamesOfCocoObject({\mdseries\slshape cgr})\index{VertexNamesOfCocoObject@\texttt{VertexNamesOfCocoObject}!for color graphs}
\label{VertexNamesOfCocoObject:for color graphs}
}\hfill{\scriptsize (operation)}}\\
\noindent\textcolor{FuncColor}{$\triangleright$\enspace\texttt{VertexNamesOfColorGraph({\mdseries\slshape cgr})\index{VertexNamesOfColorGraph@\texttt{VertexNamesOfColorGraph}!for color graphs}
\label{VertexNamesOfColorGraph:for color graphs}
}\hfill{\scriptsize (operation)}}\\


 Returns the list of names of the vertices of \mbox{\texttt{\mdseries\slshape cgr}}. Unfortunately, the more elegant name \texttt{VertexNames} is used in \textsf{Grape} as the name of a global function and can not be overloaded. 
\begin{Verbatim}[commandchars=!@|,fontsize=\small,frame=single,label=Example]
  !gapprompt@gap>| !gapinput@cgr:=JohnsonScheme(5,2);;|
  !gapprompt@gap>| !gapinput@VertexNamesOfCocoObject(cgr);|
  [ [ 1, 2 ], [ 1, 3 ], [ 1, 4 ], [ 1, 5 ], [ 2, 3 ], [ 2, 4 ], [ 2, 5 ],
  [ 3, 4 ], [ 3, 5 ], [ 4, 5 ] ]
  	    
\end{Verbatim}
 }

 

\subsection{\textcolor{Chapter }{ColorNames}}
\logpage{[ 1, 4, 4 ]}\nobreak
\hyperdef{L}{X8245B4797DD0F6AA}{}
{\noindent\textcolor{FuncColor}{$\triangleright$\enspace\texttt{ColorNames({\mdseries\slshape cgr})\index{ColorNames@\texttt{ColorNames}}
\label{ColorNames}
}\hfill{\scriptsize (operation)}}\\


 Returns the list of names of the colors of \mbox{\texttt{\mdseries\slshape cgr}}. In the following example, the color names of the Johnson scheme are the
possible cardinalities of the intersection of two $2$-element subsets of $\{1,2,3,4,5\}$. Thus loobs will get colored by $1$, since the intersection of a $2$-element set with itself will have cardinality $2$. 
\begin{Verbatim}[commandchars=!@|,fontsize=\small,frame=single,label=Example]
  !gapprompt@gap>| !gapinput@cgr:=JohnsonScheme(5,2);;|
  !gapprompt@gap>| !gapinput@ColorNames(cgr);|
  [ 2, 1, 0 ]
  	    
\end{Verbatim}
 }

 

\subsection{\textcolor{Chapter }{ArcColorOfColorGraph (first variant)}}
\logpage{[ 1, 4, 5 ]}\nobreak
\hyperdef{L}{X8530693D7D02D175}{}
{\noindent\textcolor{FuncColor}{$\triangleright$\enspace\texttt{ArcColorOfColorGraph({\mdseries\slshape cgr, u, v})\index{ArcColorOfColorGraph@\texttt{ArcColorOfColorGraph}!first variant}
\label{ArcColorOfColorGraph:first variant}
}\hfill{\scriptsize (method)}}\\
\noindent\textcolor{FuncColor}{$\triangleright$\enspace\texttt{ArcColorOfColorGraph({\mdseries\slshape cgr, arc})\index{ArcColorOfColorGraph@\texttt{ArcColorOfColorGraph}!second variant}
\label{ArcColorOfColorGraph:second variant}
}\hfill{\scriptsize (method)}}\\


 Returns the color of the arc $(u,v)$. In the second form, the arc is \mbox{\texttt{\mdseries\slshape arc}} is given as an ordered pair \texttt{[u,v]}. 
\begin{Verbatim}[commandchars=!@|,fontsize=\small,frame=single,label=Example]
  !gapprompt@gap>| !gapinput@cgr:=JohnsonScheme(5,2);;|
  !gapprompt@gap>| !gapinput@ColorNames(cgr);|
  [ 2, 1, 0 ]
  !gapprompt@gap>| !gapinput@VertexNamesOfCocoObject(cgr);|
  [ [ 1, 2 ], [ 1, 3 ], [ 1, 4 ], [ 1, 5 ], [ 2, 3 ], [ 2, 4 ], [ 2, 5 ],
  [ 3, 4 ], [ 3, 5 ], [ 4, 5 ] ]
  !gapprompt@gap>| !gapinput@ArcColorOfColorGraph(cgr,1,10);|
  3
  !gapprompt@gap>| !gapinput@ArcColorOfColorGraph(cgr,[2,9]);|
  2
  	    
\end{Verbatim}
 }

 

\subsection{\textcolor{Chapter }{ColorRepresentative}}
\logpage{[ 1, 4, 6 ]}\nobreak
\hyperdef{L}{X8121E35482D9FA8E}{}
{\noindent\textcolor{FuncColor}{$\triangleright$\enspace\texttt{ColorRepresentative({\mdseries\slshape cgr, i})\index{ColorRepresentative@\texttt{ColorRepresentative}}
\label{ColorRepresentative}
}\hfill{\scriptsize (operation)}}\\


 Returns any arc of color \mbox{\texttt{\mdseries\slshape i}} of \mbox{\texttt{\mdseries\slshape cgr}}. }

 

\subsection{\textcolor{Chapter }{Neighbors (first variant)}}
\logpage{[ 1, 4, 7 ]}\nobreak
\hyperdef{L}{X82829D3B7E0E91D1}{}
{\noindent\textcolor{FuncColor}{$\triangleright$\enspace\texttt{Neighbors({\mdseries\slshape cgr, vertices, colors})\index{Neighbors@\texttt{Neighbors}!first variant}
\label{Neighbors:first variant}
}\hfill{\scriptsize (method)}}\\
\noindent\textcolor{FuncColor}{$\triangleright$\enspace\texttt{Neighbors({\mdseries\slshape cgr, v, colors})\index{Neighbors@\texttt{Neighbors}!second variant}
\label{Neighbors:second variant}
}\hfill{\scriptsize (method)}}\\
\noindent\textcolor{FuncColor}{$\triangleright$\enspace\texttt{Neighbors({\mdseries\slshape cgr, vertices, color})\index{Neighbors@\texttt{Neighbors}!third variant}
\label{Neighbors:third variant}
}\hfill{\scriptsize (method)}}\\
\noindent\textcolor{FuncColor}{$\triangleright$\enspace\texttt{Neighbors({\mdseries\slshape cgr, v, color})\index{Neighbors@\texttt{Neighbors}!fourth variant}
\label{Neighbors:fourth variant}
}\hfill{\scriptsize (method)}}\\


 The first variant returns the set of all vertices $w$ of \mbox{\texttt{\mdseries\slshape cgr}} such that the color of the arc $(v,w)$ is an element of the set \mbox{\texttt{\mdseries\slshape colors}}, for all $v$ in \mbox{\texttt{\mdseries\slshape vertices}}. 

 The second variant gets as the second argument a single vertex of \mbox{\texttt{\mdseries\slshape cgr}}, the third gets a single color and the fourth variant gets both, a single
vertex and a single color.. 
\begin{Verbatim}[commandchars=!@|,fontsize=\small,frame=single,label=Example]
  !gapprompt@gap>| !gapinput@cgr:=JohnsonScheme(5,2);;|
  !gapprompt@gap>| !gapinput@ColorNames(cgr);|
  [ 2, 1, 0 ]
  !gapprompt@gap>| !gapinput@VertexNamesOfCocoObject(cgr);|
  [ [ 1, 2 ], [ 1, 3 ], [ 1, 4 ], [ 1, 5 ], [ 2, 3 ], [ 2, 4 ], [ 2, 5 ],
  [ 3, 4 ], [ 3, 5 ], [ 4, 5 ] ]
  !gapprompt@gap>| !gapinput@Neighbors(cgr,1,3);|
  [ 8, 9, 10 ]
  !gapprompt@gap>| !gapinput@Neighbors(cgr,1,[1,2]);|
  [ 1, 2, 3, 4, 5, 6, 7 ]
  	    
\end{Verbatim}
 }

 

\subsection{\textcolor{Chapter }{AdjacencyMatrix (first variant)}}
\logpage{[ 1, 4, 8 ]}\nobreak
\hyperdef{L}{X7D67AADE81ED3C7F}{}
{\noindent\textcolor{FuncColor}{$\triangleright$\enspace\texttt{AdjacencyMatrix({\mdseries\slshape cgr})\index{AdjacencyMatrix@\texttt{AdjacencyMatrix}!first variant}
\label{AdjacencyMatrix:first variant}
}\hfill{\scriptsize (method)}}\\
\noindent\textcolor{FuncColor}{$\triangleright$\enspace\texttt{AdjacencyMatrix({\mdseries\slshape cgr, colors})\index{AdjacencyMatrix@\texttt{AdjacencyMatrix}!second variant}
\label{AdjacencyMatrix:second variant}
}\hfill{\scriptsize (method)}}\\


 Returns the adjacency matrix of \mbox{\texttt{\mdseries\slshape cgr}}. If $A$ is the adjacency matrix of \mbox{\texttt{\mdseries\slshape cgr}}, then $A(i,j)$ is equal to the color (not to the color name!) of the arc $(i,j)$. 
\begin{Verbatim}[commandchars=!@|,fontsize=\small,frame=single,label=Example]
  !gapprompt@gap>| !gapinput@m:=[["black","red"  ,"blue" ,"blue" ,"blue" ],|
  !gapprompt@>| !gapinput@       ["blue" ,"black","red"  ,"blue" ,"blue" ],|
  !gapprompt@>| !gapinput@       ["blue" ,"blue", "black","red"  ,"blue" ],|
  !gapprompt@>| !gapinput@       ["blue" ,"blue", "blue" ,"black","red"  ],|
  !gapprompt@>| !gapinput@       ["red"  ,"blue", "blue" ,"blue" ,"black"]];;|
  !gapprompt@gap>| !gapinput@cgr:=ColorGraphByMatrix(m);|
  <color graph of order 5 and rank 3>
  !gapprompt@gap>| !gapinput@Display(AdjacencyMatrix(cgr));|
  [ [  1,  3,  2,  2,  2 ],
    [  2,  1,  3,  2,  2 ],
    [  2,  2,  1,  3,  2 ],
    [  2,  2,  2,  1,  3 ],
    [  3,  2,  2,  2,  1 ] ]
  !gapprompt@gap>| !gapinput@ColorNames(cgr)|
  [ "black", "blue", "red" ]
  	    
\end{Verbatim}
 In the second form \texttt{AdjacencyMatrix(cgr,colors)} returns a 0/1-matrix $A(i,j)$, that has entry 1 at $(i,j)$ iff the entry of \texttt{AdjacencyMatrix(cgr)} at $(i,j)$ is an element of the list \mbox{\texttt{\mdseries\slshape colors}}. 
\begin{Verbatim}[commandchars=!@|,fontsize=\small,frame=single,label=Example]
  !gapprompt@gap>| !gapinput@Display(AdjacencyMatrix(cgr, [1,3]));|
  [ [  1,  1,  0,  0,  0 ],
    [  0,  1,  1,  0,  0 ],
    [  0,  0,  1,  1,  0 ],
    [  0,  0,  0,  1,  1 ],
    [  1,  0,  0,  0,  1 ] ]
          
\end{Verbatim}
 }

 

\subsection{\textcolor{Chapter }{RowOfColorGraph}}
\logpage{[ 1, 4, 9 ]}\nobreak
\hyperdef{L}{X7AA71C1184987543}{}
{\noindent\textcolor{FuncColor}{$\triangleright$\enspace\texttt{RowOfColorGraph({\mdseries\slshape cgr, i})\index{RowOfColorGraph@\texttt{RowOfColorGraph}}
\label{RowOfColorGraph}
}\hfill{\scriptsize (operation)}}\\


 Returns the $i$-th row of the adjacency matrix of \mbox{\texttt{\mdseries\slshape cgr}} (\texttt{AdjacencyMatrix} (\ref{AdjacencyMatrix:first variant})). }

 

\subsection{\textcolor{Chapter }{ColumnOfColorGraph}}
\logpage{[ 1, 4, 10 ]}\nobreak
\hyperdef{L}{X80EEDBD98326C1DC}{}
{\noindent\textcolor{FuncColor}{$\triangleright$\enspace\texttt{ColumnOfColorGraph({\mdseries\slshape cgr, j})\index{ColumnOfColorGraph@\texttt{ColumnOfColorGraph}}
\label{ColumnOfColorGraph}
}\hfill{\scriptsize (operation)}}\\


 Returns the $j$-th column of the adjacency matrix of \mbox{\texttt{\mdseries\slshape cgr}} (\texttt{AdjacencyMatrix} (\ref{AdjacencyMatrix:first variant})). }

 

\subsection{\textcolor{Chapter }{Fibres}}
\logpage{[ 1, 4, 11 ]}\nobreak
\hyperdef{L}{X86D9A6888762649A}{}
{\noindent\textcolor{FuncColor}{$\triangleright$\enspace\texttt{Fibres({\mdseries\slshape cgr})\index{Fibres@\texttt{Fibres}}
\label{Fibres}
}\hfill{\scriptsize (operation)}}\\


 The \emph{Fibres} of a color graph are the maximal sets of vertices whose corresponding loops
all have the same color. 
\begin{Verbatim}[commandchars=!@|,fontsize=\small,frame=single,label=Example]
  !gapprompt@gap>| !gapinput@cgr:=ColorGraph(SymmetricGroup(4), Combinations([1..4]), OnSets,|
  !gapprompt@>| !gapinput@true, functions(m1,m2) return Length(Intersection(m1,m2));end);|
  <color graph of order 16 and rank 5>
  !gapprompt@gap>| !gapinput@Fibres(cgr);|
  [ [ 1 ], [ 2, 10, 14, 16 ], [ 3, 7, 9, 11, 13, 15 ], [ 4, 6, 8, 12 ], [ 5 ] ]
  !gapprompt@gap>| !gapinput@VertexNamesOfCocoObject(cgr);|
  [ [  ], [ 1 ], [ 1, 2 ], [ 1, 2, 3 ], [ 1, 2, 3, 4 ], [ 1, 2, 4 ],
  [ 1, 3 ], [ 1, 3, 4 ], [ 1, 4 ], [ 2 ], [ 2, 3 ], [ 2, 3, 4 ],
  [ 2, 4 ], [ 3 ], [ 3, 4 ], [ 4 ] ]
  	    
\end{Verbatim}
 }

 

\subsection{\textcolor{Chapter }{NumberOfFibres}}
\logpage{[ 1, 4, 12 ]}\nobreak
\hyperdef{L}{X79A514887835C4AA}{}
{\noindent\textcolor{FuncColor}{$\triangleright$\enspace\texttt{NumberOfFibres({\mdseries\slshape cgr})\index{NumberOfFibres@\texttt{NumberOfFibres}}
\label{NumberOfFibres}
}\hfill{\scriptsize (attribute)}}\\


 Returns the number of different colors of loops of \mbox{\texttt{\mdseries\slshape cgr}} (cf. \texttt{Fibres} (\ref{Fibres})). }

 

\subsection{\textcolor{Chapter }{LocalIntersectionArray}}
\logpage{[ 1, 4, 13 ]}\nobreak
\hyperdef{L}{X7D0A6D4885D48FC2}{}
{\noindent\textcolor{FuncColor}{$\triangleright$\enspace\texttt{LocalIntersectionArray({\mdseries\slshape cgr, v, w})\index{LocalIntersectionArray@\texttt{LocalIntersectionArray}}
\label{LocalIntersectionArray}
}\hfill{\scriptsize (method)}}\\
\noindent\textcolor{FuncColor}{$\triangleright$\enspace\texttt{LocalIntersectionArray({\mdseries\slshape cgr, arc})\index{LocalIntersectionArray@\texttt{LocalIntersectionArray}!alternative}
\label{LocalIntersectionArray:alternative}
}\hfill{\scriptsize (method)}}\\


 The input to this operation is a color graph \mbox{\texttt{\mdseries\slshape cgr}} and an arc. In the first version this arc is given as two parameters \mbox{\texttt{\mdseries\slshape v}}, and \mbox{\texttt{\mdseries\slshape w}}. In the second form the arc is given as ordered pair \mbox{\texttt{\mdseries\slshape arc}}. We will assume in the following that \texttt{arc=[v,w]}. The local intersection array of the arc $(v,w)$ is the square matrix $A$ of order \texttt{RankOfColorGraph(cgr)} where $A(i,j)$ is equal to the number of vertices $u$ of \mbox{\texttt{\mdseries\slshape cgr}} such that the arc $(v,u)$ has color $i$ and the arc $(u,w)$ has color $j$. 
\begin{Verbatim}[commandchars=!@|,fontsize=\small,frame=single,label=Example]
  !gapprompt@gap>| !gapinput@cgr:=JohnsonScheme(5,2);|
  <color graph of order 10 and rank 3>
  !gapprompt@gap>| !gapinput@ColorRepresentative(cgr,1);|
  [ 1, 1 ]
  !gapprompt@gap>| !gapinput@ColorRepresentative(cgr,2);|
  [ 1, 2 ]
  !gapprompt@gap>| !gapinput@ColorRepresentative(cgr,3);|
  [ 1, 8 ]
  !gapprompt@gap>| !gapinput@Display(LocalIntersectionArray(cgr,1,1));|
  [ [  1,  0,  0 ],
    [  0,  6,  0 ],
    [  0,  0,  3 ] ]
  !gapprompt@gap>| !gapinput@Display(LocalIntersectionArray(cgr,1,2));|
  [ [  0,  1,  0 ],
    [  1,  3,  2 ],
    [  0,  2,  1 ] ]
  !gapprompt@gap>| !gapinput@Display(LocalIntersectionArray(cgr,1,8));|
  [ [  0,  0,  1 ],
    [  0,  4,  2 ],
    [  1,  2,  0 ] ]
  	    
\end{Verbatim}
 }

 

\subsection{\textcolor{Chapter }{ColorMates}}
\logpage{[ 1, 4, 14 ]}\nobreak
\hyperdef{L}{X8700CA5B8386F320}{}
{\noindent\textcolor{FuncColor}{$\triangleright$\enspace\texttt{ColorMates({\mdseries\slshape cgr})\index{ColorMates@\texttt{ColorMates}}
\label{ColorMates}
}\hfill{\scriptsize (attribute)}}\\


 In a WL-stable color graph for every color $i$ there exists a color $i'$ such that whenever an arc $(u,v)$ has color $i$, then the opposite arc $(v,u)$ has color $i'$. The mapping from $i$ to $i'$ is a permutation of the colors. The function \texttt{ColorMates} returns this permutation. 
\begin{Verbatim}[commandchars=!@|,fontsize=\small,frame=single,label=Example]
  !gapprompt@gap>| !gapinput@cgr:=ColorGraph(Group((1,2,3,4,5)));;|
  !gapprompt@gap>| !gapinput@Display(AdjacencyMatrix(cgr));|
  [ [  1,  2,  3,  4,  5 ],
    [  5,  1,  2,  3,  4 ],
    [  4,  5,  1,  2,  3 ],
    [  3,  4,  5,  1,  2 ],
    [  2,  3,  4,  5,  1 ] ]
  !gapprompt@gap>| !gapinput@ColorMates(cgr);|
  (2,5)(3,4)
  	    
\end{Verbatim}
 }

 

\subsection{\textcolor{Chapter }{OutValencies (for WL-stable color graphs)}}
\logpage{[ 1, 4, 15 ]}\nobreak
\hyperdef{L}{X82C2C8757BBAEA34}{}
{\noindent\textcolor{FuncColor}{$\triangleright$\enspace\texttt{OutValencies({\mdseries\slshape cgr})\index{OutValencies@\texttt{OutValencies}!for WL-stable color graphs}
\label{OutValencies:for WL-stable color graphs}
}\hfill{\scriptsize (method)}}\\


 Let $i$ and a color of \mbox{\texttt{\mdseries\slshape cgr}}. Then there is a number $d(i)$ such that for every vertex $v$ of \mbox{\texttt{\mdseries\slshape cgr}} there is either no arc, or there are exactly $d(i)$ arcs leaving $v$. The number $d(i)$ is called the \emph{subdegree} of the color $i$. 

 The function \texttt{OutValencies} returns a the list \texttt{[d(1),d(2),...,d(RankOfColorGraph(cgr))]} }

 

\subsection{\textcolor{Chapter }{ReflexiveColors (for WL-stable color graphs)}}
\logpage{[ 1, 4, 16 ]}\nobreak
\hyperdef{L}{X7AF42EF4872960C8}{}
{\noindent\textcolor{FuncColor}{$\triangleright$\enspace\texttt{ReflexiveColors({\mdseries\slshape cgr})\index{ReflexiveColors@\texttt{ReflexiveColors}!for WL-stable color graphs}
\label{ReflexiveColors:for WL-stable color graphs}
}\hfill{\scriptsize (method)}}\\


 This function returns the list of all reflexive colors of the WL-stable color
graph \mbox{\texttt{\mdseries\slshape cgr}}. }

 }

 
\section{\textcolor{Chapter }{ Creating new (color) graphs from given color graphs}}\label{sec:cgr-functors}
\logpage{[ 1, 5, 0 ]}
\hyperdef{L}{X80FDAC6280914B37}{}
{
  

\subsection{\textcolor{Chapter }{ColorGraphByFusion}}
\logpage{[ 1, 5, 1 ]}\nobreak
\hyperdef{L}{X78832D1D7E45C6F6}{}
{\noindent\textcolor{FuncColor}{$\triangleright$\enspace\texttt{ColorGraphByFusion({\mdseries\slshape cgr, fusion})\index{ColorGraphByFusion@\texttt{ColorGraphByFusion}}
\label{ColorGraphByFusion}
}\hfill{\scriptsize (operation)}}\\


 The function takes as arguments a color graph \mbox{\texttt{\mdseries\slshape cgr}} and a fusion. The fusion can be either a list of sets of colors, or it belongs
to the category \texttt{IsFusionOfTensor} and more concretely to the family \texttt{FusionFamily(StructureConstantsOfColorGraph(cgr))}. In the latter case, \mbox{\texttt{\mdseries\slshape cgr}} has to be WL-stable. 

 The fusion-color graph has the same order like \mbox{\texttt{\mdseries\slshape cgr}}. The color of an arc $(i,j)$ in the fusion color graph is the list of all classes of \mbox{\texttt{\mdseries\slshape fusion}} to which \texttt{ArcColorOfColorGraph(cgr,i,j)} belongs. If \mbox{\texttt{\mdseries\slshape fusion}} is a partition, then the effect is that all colors in one class are fused into
the same new color. If \mbox{\texttt{\mdseries\slshape fusion}} is not a partition, then the resulting color graph will be color-isomorphic to
the fusion color graph of \mbox{\texttt{\mdseries\slshape cgr}} with respect to the coarsest partition that allows to obtain every element of \mbox{\texttt{\mdseries\slshape fusion}} as a union of classes. 
\begin{Verbatim}[commandchars=!@|,fontsize=\small,frame=single,label=Example]
  !gapprompt@gap>| !gapinput@cgr:=ColorGraph(Group((1,2,3,4,5)));|
  <color graph of order 5 and rank 5>
  !gapprompt@gap>| !gapinput@cgr2:=ColorGraphByFusion(cgr,[[1],[2,3],[4],[5]]);|
  <color graph of order 5 and rank 4>
  !gapprompt@gap>| !gapinput@Display(AdjacencyMatrix(cgr));|
  [ [  1,  2,  3,  4,  5 ],
    [  5,  1,  2,  3,  4 ],
    [  4,  5,  1,  2,  3 ],
    [  3,  4,  5,  1,  2 ],
    [  2,  3,  4,  5,  1 ] ]
  !gapprompt@gap>| !gapinput@Display(AdjacencyMatrix(cgr2));|
  [ [  1,  2,  2,  3,  4 ],
    [  4,  1,  2,  2,  3 ],
    [  3,  4,  1,  2,  2 ],
    [  2,  3,  4,  1,  2 ],
    [  2,  2,  3,  4,  1 ] ]
  !gapprompt@gap>| !gapinput@ColorNames(cgr2);|
  [ [ [ 1 ] ], [ [ 2, 3 ] ], [ [ 4 ] ], [ [ 5 ] ] ]
  	    
\end{Verbatim}
 }

 

\subsection{\textcolor{Chapter }{QuotientColorGraph}}
\logpage{[ 1, 5, 2 ]}\nobreak
\hyperdef{L}{X8038EA2E7CCAC248}{}
{\noindent\textcolor{FuncColor}{$\triangleright$\enspace\texttt{QuotientColorGraph({\mdseries\slshape cgr, part})\index{QuotientColorGraph@\texttt{QuotientColorGraph}}
\label{QuotientColorGraph}
}\hfill{\scriptsize (operation)}}\\


 \mbox{\texttt{\mdseries\slshape part}} is a partition of the vertex set of the color graph \mbox{\texttt{\mdseries\slshape cgr}} (it has to be a set of sets of vertices). The quotient graph of \mbox{\texttt{\mdseries\slshape cgr}} with respect to \mbox{\texttt{\mdseries\slshape part}} has as vertex set the classes of \mbox{\texttt{\mdseries\slshape part}}. the color of the arc $([u], [v])$ the quotient graph is the set of all colors $i$ of \mbox{\texttt{\mdseries\slshape cgr}} such that there are vertices $u'\in [u]$ and $v'\in [v]$ such that the arc $(u', v')$ has color $i$. 

 The above described color graph is also well-defined, if \mbox{\texttt{\mdseries\slshape part}} is not a partition but any set of sets of vertices of \mbox{\texttt{\mdseries\slshape cgr}}. In fact, \texttt{QuotientColorGraph} does not check, whether \mbox{\texttt{\mdseries\slshape part}} is indeed a partition. 
\begin{Verbatim}[commandchars=!@|,fontsize=\small,frame=single,label=Example]
  !gapprompt@gap>| !gapinput@s5:=SymmetricGroup(5);;|
  !gapprompt@gap>| !gapinput@cgr:=ColorGraph(s5, Arrangements([1..5],2), OnPairs,true);|
  <color graph of order 20 and rank 7>
  !gapprompt@gap>| !gapinput@part:=Set(Orbit(s5, [[1,2],[2,1]], OnSetsTuples));;|
  !gapprompt@gap>| !gapinput@part:=Set(part, x->Set(x, y->Position(VertexNamesOfCocoObject(cgr),y)));|
  [ [ 1, 5 ], [ 2, 9 ], [ 3, 13 ], [ 4, 17 ], [ 6, 10 ], [ 7, 14 ],
  [ 8, 18 ], [ 11, 15 ], [ 12, 19 ], [ 16, 20 ] ]
  !gapprompt@gap>| !gapinput@cgr2:=QuotientColorGraph(cgr,part);|
  <color graph of order 10 and rank 3>
  !gapprompt@gap>| !gapinput@ColorNames(cgr2);|
  [ [ 1, 3 ], [ 2, 4, 5, 6 ], [ 7 ] ]
  !gapprompt@gap>| !gapinput@VertexNamesOfCocoObject(cgr2);|
  [ [ 1, 5 ], [ 2, 9 ], [ 3, 13 ], [ 4, 17 ], [ 6, 10 ], [ 7, 14 ],
  [ 8, 18 ], [ 11, 15 ], [ 12, 19 ], [ 16, 20 ] ]
  	    
\end{Verbatim}
 }

 

\subsection{\textcolor{Chapter }{InducedSubColorGraph}}
\logpage{[ 1, 5, 3 ]}\nobreak
\hyperdef{L}{X7F38CEA4787B91AE}{}
{\noindent\textcolor{FuncColor}{$\triangleright$\enspace\texttt{InducedSubColorGraph({\mdseries\slshape cgr, set})\index{InducedSubColorGraph@\texttt{InducedSubColorGraph}}
\label{InducedSubColorGraph}
}\hfill{\scriptsize (operation)}}\\


 This function returns a color graph that is isomorphic to the sub color graph
induced by \mbox{\texttt{\mdseries\slshape set}}. The function that maps $i$ to \texttt{set[i]} is an embedding of the induced subgraph into \mbox{\texttt{\mdseries\slshape cgr}}. 
\begin{Verbatim}[commandchars=!@|,fontsize=\small,frame=single,label=Example]
  !gapprompt@gap>| !gapinput@cgr:=ColorGraph(SymmetricGroup(5), Combinations([1..5]), OnSets, true);|
  <color graph of order 32 and rank 56>
  !gapprompt@gap>| !gapinput@vn:=VertexNamesOfCocoObject(cgr);;|
  !gapprompt@gap>| !gapinput@fibre:=Filtered([1..Length(vn)], i->Length(vn[i])=2);|
  [ 3, 11, 15, 17, 19, 23, 25, 27, 29, 31 ]
  !gapprompt@gap>| !gapinput@cgr2:=InducedSubColorGraph(cgr,fibre);|
  <color graph of order 10 and rank 3>
  !gapprompt@gap>| !gapinput@VertexNamesOfCocoObject(cgr2);|
  [ 3, 11, 15, 17, 19, 23, 25, 27, 29, 31 ]
  	    
\end{Verbatim}
 }

 

\subsection{\textcolor{Chapter }{DirectProductColorGraphs}}
\logpage{[ 1, 5, 4 ]}\nobreak
\hyperdef{L}{X7A1C8D2979CB4C5E}{}
{\noindent\textcolor{FuncColor}{$\triangleright$\enspace\texttt{DirectProductColorGraphs({\mdseries\slshape cgr1, cgr2})\index{DirectProductColorGraphs@\texttt{DirectProductColorGraphs}}
\label{DirectProductColorGraphs}
}\hfill{\scriptsize (operation)}}\\


 Suppose, \mbox{\texttt{\mdseries\slshape cgr1}} is the color graph $(V_1,C_1,f_1)$, and \mbox{\texttt{\mdseries\slshape cgr2}} is the color graph $(V_2,C_2,f_2)$. Then the direct product of \mbox{\texttt{\mdseries\slshape cgr1}} with \mbox{\texttt{\mdseries\slshape cgr2}} has vertex set $V_1 \times V_2$, and color set $C_1 \times C_2$. The coloring function is $f_1 \times f_2$. Here $f_1 \times f_2$ acts coordinate wise. 

 The operation \texttt{DirectProductColorGraphs} returns the direct product of \mbox{\texttt{\mdseries\slshape cgr1}} with \mbox{\texttt{\mdseries\slshape cgr2}}. }

 

\subsection{\textcolor{Chapter }{WreathProductColorGraphs}}
\logpage{[ 1, 5, 5 ]}\nobreak
\hyperdef{L}{X7DD4DAEB781E3E47}{}
{\noindent\textcolor{FuncColor}{$\triangleright$\enspace\texttt{WreathProductColorGraphs({\mdseries\slshape cgr1, cgr2})\index{WreathProductColorGraphs@\texttt{WreathProductColorGraphs}}
\label{WreathProductColorGraphs}
}\hfill{\scriptsize (operation)}}\\


 Suppose, \mbox{\texttt{\mdseries\slshape cgr1}} is the color graph $(V_1,C_1,f_1)$, and \mbox{\texttt{\mdseries\slshape cgr2}} is the color graph $(V_2,C_2,f_2)$. Suppose, $D_1$ is the set of all those colors of \mbox{\texttt{\mdseries\slshape cgr1}} whose color class contains reflexive tuples. Then the wreath product of \mbox{\texttt{\mdseries\slshape cgr1}} with \mbox{\texttt{\mdseries\slshape cgr2}} has vertex set $V_1 \times V_2$. The set of colors is the union of $C_1 \times \{*\}$ with $D_1 \times C_2$. The coloring function maps pairs $((a_1,a_2), (a_1,b_2))$ to $(f_1(a_1,a_1),f_2(b_1,b_2))$, and other pairs $((a_1,a_2),(b_1,b_2))$ to $(f_1(a_1,a_2),*)$. 

 The operation \texttt{WreathProductColorGraphs} returns the wreath product of \mbox{\texttt{\mdseries\slshape cgr1}} with \mbox{\texttt{\mdseries\slshape cgr2}}. }

 

\subsection{\textcolor{Chapter }{ClosedSets (for homogeneous WL-stable color graphs)}}
\logpage{[ 1, 5, 6 ]}\nobreak
\hyperdef{L}{X879CDC86855A5C3C}{}
{\noindent\textcolor{FuncColor}{$\triangleright$\enspace\texttt{ClosedSets({\mdseries\slshape cgr})\index{ClosedSets@\texttt{ClosedSets}!for homogeneous WL-stable color graphs}
\label{ClosedSets:for homogeneous WL-stable color graphs}
}\hfill{\scriptsize (attribute)}}\\


 A set \texttt{cset} of colors of \mbox{\texttt{\mdseries\slshape cgr}} is closed if the collections of all arcs whose color is from \texttt{cset} forms an equivalence relation. This function returns a list of all closed sets
of colors of \mbox{\texttt{\mdseries\slshape cgr}}. }

 

\subsection{\textcolor{Chapter }{PartitionClosedSet (for homogeneous WL-stable color graphs)}}
\logpage{[ 1, 5, 7 ]}\nobreak
\hyperdef{L}{X7AA812417AFD425A}{}
{\noindent\textcolor{FuncColor}{$\triangleright$\enspace\texttt{PartitionClosedSet({\mdseries\slshape cgr, cset})\index{PartitionClosedSet@\texttt{PartitionClosedSet}!for homogeneous WL-stable color graphs}
\label{PartitionClosedSet:for homogeneous WL-stable color graphs}
}\hfill{\scriptsize (operation)}}\\


 A set \mbox{\texttt{\mdseries\slshape cset}} of colors of \mbox{\texttt{\mdseries\slshape cgr}} is closed if the collections of all arcs whose color is from \mbox{\texttt{\mdseries\slshape cset}} forms an equivalence relation. This function returns the vertex-partition
corresponding to this equivalence relation. It is not tested, whether \mbox{\texttt{\mdseries\slshape cset}} is indeed closed. It is required that \mbox{\texttt{\mdseries\slshape cgr}} is a homogeneous WL-stable color graph. 
\begin{Verbatim}[commandchars=!@|,fontsize=\small,frame=single,label=Example]
  !gapprompt@gap>| !gapinput@s5:=SymmetricGroup(5);;|
  !gapprompt@gap>| !gapinput@d6:=Subgroup(s5, [(1,2),(1,2,3)(4,5)]);;|
  !gapprompt@gap>| !gapinput@cgr:=ColorGraph(s5,s5,OnRight,true, function(a,b) return a*b;end);|
  <color graph of order 120 and rank 120>
  !gapprompt@gap>| !gapinput@cset:=Set(d6, x->Position(ColorNames(cgr),x));|
  [ 1, 8, 13, 24, 29, 31, 61, 68, 73, 84, 89, 91 ]
  !gapprompt@gap>| !gapinput@IsWLStableColorGraph(cgr);|
  true
  !gapprompt@gap>| !gapinput@IsHomogeneous(cgr);|
  true
  !gapprompt@gap>| !gapinput@part:=PartitionClosedSet(cgr,cset);;|
  !gapprompt@gap>| !gapinput@cgr2:=QuotientColorGraph(cgr,part);|
  <color graph of order 10 and rank 3>
  	    
\end{Verbatim}
 }

 

\subsection{\textcolor{Chapter }{BaseGraphOfColorGraph (first variant)}}
\logpage{[ 1, 5, 8 ]}\nobreak
\hyperdef{L}{X7A18F2757CB7B69C}{}
{\noindent\textcolor{FuncColor}{$\triangleright$\enspace\texttt{BaseGraphOfColorGraph({\mdseries\slshape cgr, color})\index{BaseGraphOfColorGraph@\texttt{BaseGraphOfColorGraph}!first variant}
\label{BaseGraphOfColorGraph:first variant}
}\hfill{\scriptsize (method)}}\\
\noindent\textcolor{FuncColor}{$\triangleright$\enspace\texttt{BaseGraphOfColorGraph({\mdseries\slshape cgr, cset})\index{BaseGraphOfColorGraph@\texttt{BaseGraphOfColorGraph}!second variant}
\label{BaseGraphOfColorGraph:second variant}
}\hfill{\scriptsize (method)}}\\


 This function extracts graphs from a color graph. In the first variant, the
second argument is one color. In this case the digraph with vertex set \texttt{[1..OrderOfColorGraph(cgr)]} and with all arcs of color \mbox{\texttt{\mdseries\slshape color}} from \mbox{\texttt{\mdseries\slshape cgr}}. 

 In the second case the arc-set of the result consists of all arcs with color
from \mbox{\texttt{\mdseries\slshape cset}} of \mbox{\texttt{\mdseries\slshape cgr}}. 

 This function is available only if \textsf{Grape} is loaded. 
\begin{Verbatim}[commandchars=!@|,fontsize=\small,frame=single,label=Example]
  !gapprompt@gap>| !gapinput@cgr:=JohnsonScheme(5,2);|
  <color graph of order 10 and rank 3>
  !gapprompt@gap>| !gapinput@OutValencies(cgr);|
  [ 1, 6, 3 ]
  !gapprompt@gap>| !gapinput@gamma:=BaseGraphOfColorGraph(cgr,3);;|
  !gapprompt@gap>| !gapinput@IsDistanceRegular(gamma);|
  true
  !gapprompt@gap>| !gapinput@GlobalParameters(gamma);|
  [ [ 0, 0, 3 ], [ 1, 0, 2 ], [ 1, 2, 0 ] ]
  	    
\end{Verbatim}
 }

 }

 
\section{\textcolor{Chapter }{ Testing properties of color graphs }}\label{Sec:cgr-properties}
\logpage{[ 1, 6, 0 ]}
\hyperdef{L}{X82904EE47E50D920}{}
{
  

\subsection{\textcolor{Chapter }{IsUndirectedColorGraph}}
\logpage{[ 1, 6, 1 ]}\nobreak
\hyperdef{L}{X7ABD36357BBD6F03}{}
{\noindent\textcolor{FuncColor}{$\triangleright$\enspace\texttt{IsUndirectedColorGraph({\mdseries\slshape cgr})\index{IsUndirectedColorGraph@\texttt{IsUndirectedColorGraph}}
\label{IsUndirectedColorGraph}
}\hfill{\scriptsize (property)}}\\


 A color graph is called \emph{undirected} if for all vertices $u$ and $v$ the arc $(u,v)$ has the same color as the arc $(v,u)$. The function tests this property for \mbox{\texttt{\mdseries\slshape cgr}}. 
\begin{Verbatim}[commandchars=!@|,fontsize=\small,frame=single,label=Example]
  !gapprompt@gap>| !gapinput@cgr:=ColorGraph(Group((1,2,3,4,5)));|
  <color graph of order 5 and rank 5>
  !gapprompt@gap>| !gapinput@IsUndirectedColorGraph(cgr);|
  false
  !gapprompt@gap>| !gapinput@ArcColorOfColorGraph(cgr,[1,2]);|
  2
  !gapprompt@gap>| !gapinput@ArcColorOfColorGraph(cgr,[2,1]);|
  5
  !gapprompt@gap>| !gapinput@cgr2:=ColorGraphByFusion(cgr, [[1],[2,5],[3,4]]);|
  <color graph of order 5 and rank 3>
  !gapprompt@gap>| !gapinput@IsUndirectedColorGraph(cgr2);|
  true
  	    
\end{Verbatim}
 }

 

\subsection{\textcolor{Chapter }{IsHomogeneous}}
\logpage{[ 1, 6, 2 ]}\nobreak
\hyperdef{L}{X7D8EA4528082A74D}{}
{\noindent\textcolor{FuncColor}{$\triangleright$\enspace\texttt{IsHomogeneous({\mdseries\slshape cgr})\index{IsHomogeneous@\texttt{IsHomogeneous}}
\label{IsHomogeneous}
}\hfill{\scriptsize (property)}}\\


 A color graph is homogeneous if it has just one fibre. For a WL-stable color
graph this means that it has just one reflexive color. For a WL-stable color
graph this means that it correspond to an association scheme. 
\begin{Verbatim}[commandchars=!@|,fontsize=\small,frame=single,label=Example]
  !gapprompt@gap>| !gapinput@e8:=ElementaryAbelianGroup(8);|
  <pc group of size 8 with 3 generators>
  !gapprompt@gap>| !gapinput@e8:=Action(e8,AsList(e8), OnRight);|
  Group([ (1,2)(3,5)(4,6)(7,8), (1,3)(2,5)(4,7)(6,8), (1,4)(2,6)(3,7)(5,8) ])
  !gapprompt@gap>| !gapinput@cgr:=ColorGraph(e8,Combinations([1..DegreeAction(g)],2), OnSets);|
  <color graph of order 28 and rank 112>
  !gapprompt@gap>| !gapinput@IsHomogeneous(cgr);|
  false
  	    
\end{Verbatim}
 }

 

\subsection{\textcolor{Chapter }{IsWLStableColorGraph}}
\logpage{[ 1, 6, 3 ]}\nobreak
\hyperdef{L}{X82B217447E358C43}{}
{\noindent\textcolor{FuncColor}{$\triangleright$\enspace\texttt{IsWLStableColorGraph({\mdseries\slshape cgr})\index{IsWLStableColorGraph@\texttt{IsWLStableColorGraph}}
\label{IsWLStableColorGraph}
}\hfill{\scriptsize (property)}}\\


 This function returns true if \mbox{\texttt{\mdseries\slshape cgr}} is stable under the Weisfeiler-Leman algorithm, that is, whether it is the
color graph of a coherent configuration. 
\begin{Verbatim}[commandchars=!@|,fontsize=\small,frame=single,label=Example]
  !gapprompt@gap>| !gapinput@cgr:=ColorGraph(Center(GL(2,7)), GF(7)^2, OnRight, true,|
  !gapprompt@>| !gapinput@function(a,b) return NormedRowVector(a-b);end);|
  <color graph of order 49 and rank 9>
  !gapprompt@gap>| !gapinput@IsWLStableColorGraph(cgr);|
  true
  	    
\end{Verbatim}
 }

 

\subsection{\textcolor{Chapter }{IsSchurian}}
\logpage{[ 1, 6, 4 ]}\nobreak
\hyperdef{L}{X7EC0451D798802B9}{}
{\noindent\textcolor{FuncColor}{$\triangleright$\enspace\texttt{IsSchurian({\mdseries\slshape cgr})\index{IsSchurian@\texttt{IsSchurian}}
\label{IsSchurian}
}\hfill{\scriptsize (property)}}\\


 A color graph is called \emph{Schurian} if it is color isomorphic to the color graph of orbitals of its automorphism
group. 
\begin{Verbatim}[commandchars=!@|,fontsize=\small,frame=single,label=Example]
  !gapprompt@gap>| !gapinput@lcgr:=AllAssociationSchemes(15);;|
  !gapprompt@gap>| !gapinput@lcgr:=Filtered(lcgr, x->not IsSchurian(x));|
  [ AS(15,5) ]
  	    
\end{Verbatim}
 }

 

\subsection{\textcolor{Chapter }{IsPrimitive (for WL-stable color graphs)}}
\logpage{[ 1, 6, 5 ]}\nobreak
\hyperdef{L}{X7DEB50307A5353B5}{}
{\noindent\textcolor{FuncColor}{$\triangleright$\enspace\texttt{IsPrimitive({\mdseries\slshape cgr})\index{IsPrimitive@\texttt{IsPrimitive}!for WL-stable color graphs}
\label{IsPrimitive:for WL-stable color graphs}
}\hfill{\scriptsize (property)}}\\


 A WL-stable color graph is primitive if all its loopless base graphs are
strongly connected (cf. \texttt{BaseGraphOfColorGraph} (\ref{BaseGraphOfColorGraph:first variant})). This function tests, whether \mbox{\texttt{\mdseries\slshape cgr}} is primitive or not. 
\begin{Verbatim}[commandchars=!@|,fontsize=\small,frame=single,label=Example]
  !gapprompt@gap>| !gapinput@cgr:=ColorGraph(Group((1,2,3,4)));|
  <color graph of order 4 and rank 4>
  !gapprompt@gap>| !gapinput@IsPrimitive(cgr);|
  false
  !gapprompt@gap>| !gapinput@ReflexiveColors(cgr);|
  [ 1 ]
  !gapprompt@gap>| !gapinput@IsConnectedGraph(BaseGraphOfColorGraph(cgr,2));|
  true
  !gapprompt@gap>| !gapinput@IsConnectedGraph(BaseGraphOfColorGraph(cgr,3));|
  false
  !gapprompt@gap>| !gapinput@IsConnectedGraph(BaseGraphOfColorGraph(cgr,4));|
  true
  	    
\end{Verbatim}
 }

 }

 
\section{\textcolor{Chapter }{Symmetries of color graphs}}\logpage{[ 1, 7, 0 ]}
\hyperdef{L}{X82E1845978248868}{}
{
  

\subsection{\textcolor{Chapter }{KnownGroupOfAutomorphisms (for color graphs)}}
\logpage{[ 1, 7, 1 ]}\nobreak
\hyperdef{L}{X7EC9E8857F29742B}{}
{\noindent\textcolor{FuncColor}{$\triangleright$\enspace\texttt{KnownGroupOfAutomorphisms({\mdseries\slshape cgr})\index{KnownGroupOfAutomorphisms@\texttt{KnownGroupOfAutomorphisms}!for color graphs}
\label{KnownGroupOfAutomorphisms:for color graphs}
}\hfill{\scriptsize (operation)}}\\


 This function returns the group of all automorphisms of \mbox{\texttt{\mdseries\slshape cgr}} that \textsf{COCO2P} knows at the given moment. }

 

\subsection{\textcolor{Chapter }{AutGroupOfCocoObject (for color graphs)}}
\logpage{[ 1, 7, 2 ]}\nobreak
\hyperdef{L}{X7ED15B0F7C276AEB}{}
{\noindent\textcolor{FuncColor}{$\triangleright$\enspace\texttt{AutGroupOfCocoObject({\mdseries\slshape cgr})\index{AutGroupOfCocoObject@\texttt{AutGroupOfCocoObject}!for color graphs}
\label{AutGroupOfCocoObject:for color graphs}
}\hfill{\scriptsize (attribute)}}\\
\noindent\textcolor{FuncColor}{$\triangleright$\enspace\texttt{AutomorphismGroup({\mdseries\slshape cgr})\index{AutomorphismGroup@\texttt{AutomorphismGroup}!for color graphs}
\label{AutomorphismGroup:for color graphs}
}\hfill{\scriptsize (method)}}\\


 Returns the group of all permutations of the vertices of \mbox{\texttt{\mdseries\slshape cgr}} that preserve the color of all arcs. }

 

\subsection{\textcolor{Chapter }{IsAutomorphismOfObject (for color graphs)}}
\logpage{[ 1, 7, 3 ]}\nobreak
\hyperdef{L}{X7FEA59547C68E12D}{}
{\noindent\textcolor{FuncColor}{$\triangleright$\enspace\texttt{IsAutomorphismOfObject({\mdseries\slshape cgr, perm})\index{IsAutomorphismOfObject@\texttt{IsAutomorphismOfObject}!for color graphs}
\label{IsAutomorphismOfObject:for color graphs}
}\hfill{\scriptsize (operation)}}\\
\noindent\textcolor{FuncColor}{$\triangleright$\enspace\texttt{IsAutomorphismOfColorGraph({\mdseries\slshape cgr, perm})\index{IsAutomorphismOfColorGraph@\texttt{IsAutomorphismOfColorGraph}!for color graphs}
\label{IsAutomorphismOfColorGraph:for color graphs}
}\hfill{\scriptsize (operation)}}\\


 Returns \texttt{true}, if \mbox{\texttt{\mdseries\slshape perm}} is an automorphism of \mbox{\texttt{\mdseries\slshape cgr}}. In that case \textsf{COCO2P} adds \mbox{\texttt{\mdseries\slshape perm}} to the known automorphisms of \mbox{\texttt{\mdseries\slshape cgr}}. }

 

\subsection{\textcolor{Chapter }{IsomorphismCocoObjects (for color graphs)}}
\logpage{[ 1, 7, 4 ]}\nobreak
\hyperdef{L}{X7E72DC9C81E1FC86}{}
{\noindent\textcolor{FuncColor}{$\triangleright$\enspace\texttt{IsomorphismCocoObjects({\mdseries\slshape cgr1, cgr2})\index{IsomorphismCocoObjects@\texttt{IsomorphismCocoObjects}!for color graphs}
\label{IsomorphismCocoObjects:for color graphs}
}\hfill{\scriptsize (operation)}}\\
\noindent\textcolor{FuncColor}{$\triangleright$\enspace\texttt{IsomorphismColorGraphs({\mdseries\slshape cgr1, cgr2})\index{IsomorphismColorGraphs@\texttt{IsomorphismColorGraphs}!for color graphs}
\label{IsomorphismColorGraphs:for color graphs}
}\hfill{\scriptsize (operation)}}\\


 An isomorphism from \mbox{\texttt{\mdseries\slshape cgr1}} to \mbox{\texttt{\mdseries\slshape cgr2}} is a bijection between the vertex sets that preserves the color of arcs
(including the names of colors). 

 This operation returns an isomorphism from \mbox{\texttt{\mdseries\slshape cgr1}} to \mbox{\texttt{\mdseries\slshape cgr2}} if it exists, and \texttt{fail} if it does not exists. }

 

\subsection{\textcolor{Chapter }{IsIsomorphicCocoObject (for color graphs)}}
\logpage{[ 1, 7, 5 ]}\nobreak
\hyperdef{L}{X848658A983BFE358}{}
{\noindent\textcolor{FuncColor}{$\triangleright$\enspace\texttt{IsIsomorphicCocoObject({\mdseries\slshape cgr1, cgr2})\index{IsIsomorphicCocoObject@\texttt{IsIsomorphicCocoObject}!for color graphs}
\label{IsIsomorphicCocoObject:for color graphs}
}\hfill{\scriptsize (operation)}}\\
\noindent\textcolor{FuncColor}{$\triangleright$\enspace\texttt{IsIsomorphicColorGraph({\mdseries\slshape cgr1, cgr2})\index{IsIsomorphicColorGraph@\texttt{IsIsomorphicColorGraph}!for color graphs}
\label{IsIsomorphicColorGraph:for color graphs}
}\hfill{\scriptsize (operation)}}\\


 Returns \texttt{true} if \mbox{\texttt{\mdseries\slshape cgr1}} and \mbox{\texttt{\mdseries\slshape cgr2}} are isomorphic, and \texttt{false} otherwise (cf. \texttt{IsomorphismCocoObjects} (\ref{IsomorphismCocoObjects:for color graphs})) }

 

\subsection{\textcolor{Chapter }{IsIsomorphismOfObjects (for color graphs)}}
\logpage{[ 1, 7, 6 ]}\nobreak
\hyperdef{L}{X821668C487B0EC6B}{}
{\noindent\textcolor{FuncColor}{$\triangleright$\enspace\texttt{IsIsomorphismOfObjects({\mdseries\slshape cgr1, cgr2, g})\index{IsIsomorphismOfObjects@\texttt{IsIsomorphismOfObjects}!for color graphs}
\label{IsIsomorphismOfObjects:for color graphs}
}\hfill{\scriptsize (operation)}}\\
\noindent\textcolor{FuncColor}{$\triangleright$\enspace\texttt{IsIsomorphismOfColorGraphs({\mdseries\slshape cgr1, cgr2, g})\index{IsIsomorphismOfColorGraphs@\texttt{IsIsomorphismOfColorGraphs}!for color graphs}
\label{IsIsomorphismOfColorGraphs:for color graphs}
}\hfill{\scriptsize (operation)}}\\


 Returns \texttt{true} if \mbox{\texttt{\mdseries\slshape g}} is an isomorphism fro \mbox{\texttt{\mdseries\slshape cgr1}} to \mbox{\texttt{\mdseries\slshape cgr2}}, and \texttt{false} otherwise (cf. \texttt{IsomorphismCocoObjects} (\ref{IsomorphismCocoObjects:for color graphs})). }

 

\subsection{\textcolor{Chapter }{KnownGroupOfColorAutomorphisms}}
\logpage{[ 1, 7, 7 ]}\nobreak
\hyperdef{L}{X82ACB834799F31B4}{}
{\noindent\textcolor{FuncColor}{$\triangleright$\enspace\texttt{KnownGroupOfColorAutomorphisms({\mdseries\slshape cgr})\index{KnownGroupOfColorAutomorphisms@\texttt{KnownGroupOfColorAutomorphisms}}
\label{KnownGroupOfColorAutomorphisms}
}\hfill{\scriptsize (operation)}}\\


 This function returns the group of all color automorphisms of \mbox{\texttt{\mdseries\slshape cgr}} that \textsf{COCO2P} knows at the given moment. }

 

\subsection{\textcolor{Chapter }{KnownGroupOfColorAutomorphismsOnColors}}
\logpage{[ 1, 7, 8 ]}\nobreak
\hyperdef{L}{X850C31007C88595D}{}
{\noindent\textcolor{FuncColor}{$\triangleright$\enspace\texttt{KnownGroupOfColorAutomorphismsOnColors({\mdseries\slshape cgr})\index{KnownGroupOfColorAutomorphismsOnColors@\texttt{Known}\-\texttt{Group}\-\texttt{Of}\-\texttt{Color}\-\texttt{Automorphisms}\-\texttt{On}\-\texttt{Colors}}
\label{KnownGroupOfColorAutomorphismsOnColors}
}\hfill{\scriptsize (operation)}}\\


 This function returns the group of all color automorphisms of \mbox{\texttt{\mdseries\slshape cgr}} that \textsf{COCO2P} knows at the given moment, acting on the colors of \mbox{\texttt{\mdseries\slshape cgr}}. }

 

\subsection{\textcolor{Chapter }{LiftToColorAutomorphism}}
\logpage{[ 1, 7, 9 ]}\nobreak
\hyperdef{L}{X85B8184C834A3863}{}
{\noindent\textcolor{FuncColor}{$\triangleright$\enspace\texttt{LiftToColorAutomorphism({\mdseries\slshape cgr, perm})\index{LiftToColorAutomorphism@\texttt{LiftToColorAutomorphism}}
\label{LiftToColorAutomorphism}
}\hfill{\scriptsize (operation)}}\\


 \mbox{\texttt{\mdseries\slshape cgr}} is a color graph and \mbox{\texttt{\mdseries\slshape perm}} is a permutation of its colors. The function constructs a color automorphism
of \mbox{\texttt{\mdseries\slshape cgr}} that acts like \mbox{\texttt{\mdseries\slshape perm}} on the colors. If such a color automorphism does not exist, then \texttt{fail} is returned. 

 If \mbox{\texttt{\mdseries\slshape perm}} is liftable, then the result of the lifting is added to the known group of
color automorphisms of \mbox{\texttt{\mdseries\slshape cgr}}. }

 

\subsection{\textcolor{Chapter }{LiftToColorIsomorphism}}
\logpage{[ 1, 7, 10 ]}\nobreak
\hyperdef{L}{X8081D1EF7A47A665}{}
{\noindent\textcolor{FuncColor}{$\triangleright$\enspace\texttt{LiftToColorIsomorphism({\mdseries\slshape cgr1, cgr2, ciso})\index{LiftToColorIsomorphism@\texttt{LiftToColorIsomorphism}}
\label{LiftToColorIsomorphism}
}\hfill{\scriptsize (operation)}}\\


 \mbox{\texttt{\mdseries\slshape cgr1}} and \mbox{\texttt{\mdseries\slshape cgr2}} are color graphs of the same rank, and \mbox{\texttt{\mdseries\slshape ciso}} is a bijection from the colors of \mbox{\texttt{\mdseries\slshape cgr1}} to the colors of \mbox{\texttt{\mdseries\slshape cgr2}}. The function constructs a color isomorphism from \mbox{\texttt{\mdseries\slshape cgr1}} to \mbox{\texttt{\mdseries\slshape cgr2}} that acts like \mbox{\texttt{\mdseries\slshape ciso}} on the colors. If such a color isomorphism does not exist, then \texttt{fail} is returned. }

 

\subsection{\textcolor{Chapter }{ColorIsomorphismColorGraphs}}
\logpage{[ 1, 7, 11 ]}\nobreak
\hyperdef{L}{X7D5857C385A376FE}{}
{\noindent\textcolor{FuncColor}{$\triangleright$\enspace\texttt{ColorIsomorphismColorGraphs({\mdseries\slshape cgr1, cgr2})\index{ColorIsomorphismColorGraphs@\texttt{ColorIsomorphismColorGraphs}}
\label{ColorIsomorphismColorGraphs}
}\hfill{\scriptsize (operation)}}\\


 This operation returns a color isomorphism from \mbox{\texttt{\mdseries\slshape cgr1}} to \mbox{\texttt{\mdseries\slshape cgr2}} if it exists, and \texttt{fail} otherwise. 

 Here a color isomorphism is an ordered pair \texttt{[f,g]}, where \texttt{f} is a bijection from \texttt{[1..Order(cgr1)]} to \texttt{[1..Order(cgr2)]} and where \texttt{g} is a bijection from \texttt{[1..Rank(cgr1)]} to \texttt{[1..Rank(cgr2)]}, such that \texttt{ArcColorOfColorGraph(cgr1,u,v)\texttt{\symbol{94}}g =
ArcColorOfColorGraph(cgr2,u\texttt{\symbol{94}}f,v\texttt{\symbol{94}}v)}, for all \texttt{u} and \texttt{v} from \texttt{[1..Order(cgr1)]}. 

 At the moment, this operation is implemented only for WL-stable color graphs. }

 

\subsection{\textcolor{Chapter }{IsColorIsomorphicColorGraph}}
\logpage{[ 1, 7, 12 ]}\nobreak
\hyperdef{L}{X7FAD8F0F79023E02}{}
{\noindent\textcolor{FuncColor}{$\triangleright$\enspace\texttt{IsColorIsomorphicColorGraph({\mdseries\slshape cgr1, cgr2})\index{IsColorIsomorphicColorGraph@\texttt{IsColorIsomorphicColorGraph}}
\label{IsColorIsomorphicColorGraph}
}\hfill{\scriptsize (operation)}}\\


 This operation returns \texttt{true} if \mbox{\texttt{\mdseries\slshape cgr1}} and \mbox{\texttt{\mdseries\slshape cgr2}} are color isomorphic, and \texttt{false} otherwise. 

 At the moment, this operation is implemented only from WL-stable color graphs. }

 

\subsection{\textcolor{Chapter }{IsColorIsomorphismOfColorGraphs (for color graphs)}}
\logpage{[ 1, 7, 13 ]}\nobreak
\hyperdef{L}{X878BF5EA8388B9D9}{}
{\noindent\textcolor{FuncColor}{$\triangleright$\enspace\texttt{IsColorIsomorphismOfColorGraphs({\mdseries\slshape cgr1, cgr2, g, h})\index{IsColorIsomorphismOfColorGraphs@\texttt{IsColorIsomorphismOfColorGraphs}!for color graphs}
\label{IsColorIsomorphismOfColorGraphs:for color graphs}
}\hfill{\scriptsize (operation)}}\\
\noindent\textcolor{FuncColor}{$\triangleright$\enspace\texttt{IsColorIsomorphismOfColorGraphs({\mdseries\slshape cgr1, cgr2[, g, h]})\index{IsColorIsomorphismOfColorGraphs@\texttt{IsColorIsomorphismOfColorGraphs}}
\label{IsColorIsomorphismOfColorGraphs}
}\hfill{\scriptsize (operation)}}\\


 This operation returns \texttt{true} if \mbox{\texttt{\mdseries\slshape [g,h]}} is a colorisomorphism from \mbox{\texttt{\mdseries\slshape cgr1}} to \mbox{\texttt{\mdseries\slshape cgr2}} (see \texttt{ColorIsomorphismColorGraphs} (\ref{ColorIsomorphismColorGraphs}) for a definition of color isomorphisms). }

 

\subsection{\textcolor{Chapter }{ColorAutomorphismGroup}}
\logpage{[ 1, 7, 14 ]}\nobreak
\hyperdef{L}{X82A68C3F81741249}{}
{\noindent\textcolor{FuncColor}{$\triangleright$\enspace\texttt{ColorAutomorphismGroup({\mdseries\slshape cgr})\index{ColorAutomorphismGroup@\texttt{ColorAutomorphismGroup}}
\label{ColorAutomorphismGroup}
}\hfill{\scriptsize (attribute)}}\\


 This function computes and returns the color automorphism group of \mbox{\texttt{\mdseries\slshape cgr}}. This group consists of all permutations of the vertices of the color graph,
that map arcs of the same color to arcs of the same color. In particular, it
may act non-trivially on the colors of \mbox{\texttt{\mdseries\slshape cgr}}. 

 If \mbox{\texttt{\mdseries\slshape cgr}} is a Schurian WL-stable color graph, then its color automorphism group is
equal to the normalizer of its automorphism group in the full symmetric group
of the vertices of \mbox{\texttt{\mdseries\slshape cgr}}. In some (rare) cases, this way to compute normalizers can be quicker than
the built-in gap-functions. 

 At the moment, this function is implemented only for WL-stable color graphs. }

 

\subsection{\textcolor{Chapter }{ColorAutomorphismGroupOnColors}}
\logpage{[ 1, 7, 15 ]}\nobreak
\hyperdef{L}{X7D339D77856CE298}{}
{\noindent\textcolor{FuncColor}{$\triangleright$\enspace\texttt{ColorAutomorphismGroupOnColors({\mdseries\slshape cgr})\index{ColorAutomorphismGroupOnColors@\texttt{ColorAutomorphismGroupOnColors}}
\label{ColorAutomorphismGroupOnColors}
}\hfill{\scriptsize (attribute)}}\\


 The color automorphism of \mbox{\texttt{\mdseries\slshape cgr}} acts on the colors of \mbox{\texttt{\mdseries\slshape cgr}} with the automorphism group of \mbox{\texttt{\mdseries\slshape cgr}} as kernel. This function computes and returns this action. 

 At the moment, this function is implemented only for WL-stable color graphs. }

 

\subsection{\textcolor{Chapter }{KnownGroupOfAlgebraicAutomorphisms}}
\logpage{[ 1, 7, 16 ]}\nobreak
\hyperdef{L}{X7FD158627A750218}{}
{\noindent\textcolor{FuncColor}{$\triangleright$\enspace\texttt{KnownGroupOfAlgebraicAutomorphisms({\mdseries\slshape cgr})\index{KnownGroupOfAlgebraicAutomorphisms@\texttt{KnownGroupOfAlgebraicAutomorphisms}}
\label{KnownGroupOfAlgebraicAutomorphisms}
}\hfill{\scriptsize (operation)}}\\


 This function returns the group of all algebraic automorphisms of \mbox{\texttt{\mdseries\slshape cgr}} that \textsf{COCO2P} knows at the given moment. }

 

\subsection{\textcolor{Chapter }{AlgebraicAutomorphismGroup}}
\logpage{[ 1, 7, 17 ]}\nobreak
\hyperdef{L}{X7C10A5D37F3DDAD5}{}
{\noindent\textcolor{FuncColor}{$\triangleright$\enspace\texttt{AlgebraicAutomorphismGroup({\mdseries\slshape cgr})\index{AlgebraicAutomorphismGroup@\texttt{AlgebraicAutomorphismGroup}}
\label{AlgebraicAutomorphismGroup}
}\hfill{\scriptsize (attribute)}}\\


 The algebraic automorphism group of a WL-stable color graph is nothing but the
automorphism group of its tensor of structure constants. The color
automorphism group in its action on colors embeds naturally into the algebraic
automorphism group. }

 }

 }

 
\chapter{\textcolor{Chapter }{Structure Constants Tensors}}\logpage{[ 2, 0, 0 ]}
\hyperdef{L}{X79EEFFAF7BB75DEA}{}
{
  
\section{\textcolor{Chapter }{Introduction}}\logpage{[ 2, 1, 0 ]}
\hyperdef{L}{X7DFB63A97E67C0A1}{}
{
  \textsf{COCO2P} introduces its own data-type for structure constants tensors of coherent
algebras. The methods provided by \textsf{COCO2P} are tailored for this use. The emphasis lies on symmetries, quotients (by
closed sets) and mergings (fusions). }

 
\section{\textcolor{Chapter }{ Functions for the construction of tensors }}\logpage{[ 2, 2, 0 ]}
\hyperdef{L}{X862A696D7905A540}{}
{
  

\subsection{\textcolor{Chapter }{StructureConstantsOfColorGraph}}
\logpage{[ 2, 2, 1 ]}\nobreak
\hyperdef{L}{X81048C117AA4BA48}{}
{\noindent\textcolor{FuncColor}{$\triangleright$\enspace\texttt{StructureConstantsOfColorGraph({\mdseries\slshape cgr})\index{StructureConstantsOfColorGraph@\texttt{StructureConstantsOfColorGraph}}
\label{StructureConstantsOfColorGraph}
}\hfill{\scriptsize (attribute)}}\\


 This function expects a WL-stable color graph \mbox{\texttt{\mdseries\slshape cgr}}, and computes its tensor of structure constants. The result is the structure
constants tensor $T$ of \mbox{\texttt{\mdseries\slshape cgr}}. This object encodes a third-order tensor. For every color $k$ of \mbox{\texttt{\mdseries\slshape cgr}}, the matrix $T(i,j,k)$ is equal to the \texttt{LocalIntersectionArray} (\ref{LocalIntersectionArray}) of any arc of color $k$ in \mbox{\texttt{\mdseries\slshape cgr}}. }

 

\subsection{\textcolor{Chapter }{DenseTensorFromEntries}}
\logpage{[ 2, 2, 2 ]}\nobreak
\hyperdef{L}{X8683BAE181ED9E76}{}
{\noindent\textcolor{FuncColor}{$\triangleright$\enspace\texttt{DenseTensorFromEntries({\mdseries\slshape entries})\index{DenseTensorFromEntries@\texttt{DenseTensorFromEntries}}
\label{DenseTensorFromEntries}
}\hfill{\scriptsize (function)}}\\


 The argument \mbox{\texttt{\mdseries\slshape entries}} is a list of lists of lists of integers. There has to be a number $n$ such that \texttt{Length(entries)=n}, for all $1\le i,j \le n$ \texttt{Length(entries[i])=n}, and \texttt{Length(entries[i][j]=n}. Otherwise there are no restrictions. 

 The function returns the tensor-object for \mbox{\texttt{\mdseries\slshape entries}}. 

 Note that this function does not check, whether the entries are integers or
even numbers. One can also view the datatype of tensors as a type that encodes
complete colored hyper-graphs with hyper-arcs of length $3$. Even though there is not much infrastructure implemented in \textsf{COCO2P} for such objects, at least it is possible to check isomorphism and to compute
automorphism groups. }

 }

 
\section{\textcolor{Chapter }{Functions for the inspection of tensors}}\logpage{[ 2, 3, 0 ]}
\hyperdef{L}{X7CFD172A870557C5}{}
{
  

\subsection{\textcolor{Chapter }{OrderOfTensor}}
\logpage{[ 2, 3, 1 ]}\nobreak
\hyperdef{L}{X791D1A5F78FC32CC}{}
{\noindent\textcolor{FuncColor}{$\triangleright$\enspace\texttt{OrderOfTensor({\mdseries\slshape tensor})\index{OrderOfTensor@\texttt{OrderOfTensor}}
\label{OrderOfTensor}
}\hfill{\scriptsize (attribute)}}\\
\noindent\textcolor{FuncColor}{$\triangleright$\enspace\texttt{OrderOfCocoObject({\mdseries\slshape tensor})\index{OrderOfCocoObject@\texttt{OrderOfCocoObject}!for tensors}
\label{OrderOfCocoObject:for tensors}
}\hfill{\scriptsize (attribute)}}\\
\noindent\textcolor{FuncColor}{$\triangleright$\enspace\texttt{Order({\mdseries\slshape tensor})\index{Order@\texttt{Order}!for tensors}
\label{Order:for tensors}
}\hfill{\scriptsize (attribute)}}\\


 Returns the order of the tensor. If it is equal to $n$ then this means that \mbox{\texttt{\mdseries\slshape tensor}} is an $n\times n \times n$-array. }

 

\subsection{\textcolor{Chapter }{VertexNamesOfCocoObject (for tensors)}}
\logpage{[ 2, 3, 2 ]}\nobreak
\hyperdef{L}{X79CBAB5483A0BD8A}{}
{\noindent\textcolor{FuncColor}{$\triangleright$\enspace\texttt{VertexNamesOfCocoObject({\mdseries\slshape tensor})\index{VertexNamesOfCocoObject@\texttt{VertexNamesOfCocoObject}!for tensors}
\label{VertexNamesOfCocoObject:for tensors}
}\hfill{\scriptsize (operation)}}\\
\noindent\textcolor{FuncColor}{$\triangleright$\enspace\texttt{VertexNamesOfTensor({\mdseries\slshape tensor})\index{VertexNamesOfTensor@\texttt{VertexNamesOfTensor}!for tensors}
\label{VertexNamesOfTensor:for tensors}
}\hfill{\scriptsize (operation)}}\\


 Returns the list of names of the vertices of \mbox{\texttt{\mdseries\slshape tensor}}. }

 

\subsection{\textcolor{Chapter }{EntryOfTensor}}
\logpage{[ 2, 3, 3 ]}\nobreak
\hyperdef{L}{X84B4DA3282F5965C}{}
{\noindent\textcolor{FuncColor}{$\triangleright$\enspace\texttt{EntryOfTensor({\mdseries\slshape tensor, i, j, k})\index{EntryOfTensor@\texttt{EntryOfTensor}}
\label{EntryOfTensor}
}\hfill{\scriptsize (operation)}}\\


 Returns the entry at index $(i,j,k)$ of \mbox{\texttt{\mdseries\slshape tensor}}. }

 

\subsection{\textcolor{Chapter }{ReflexiveColors (for structure constants tensors)}}
\logpage{[ 2, 3, 4 ]}\nobreak
\hyperdef{L}{X854662A2783CDD3A}{}
{\noindent\textcolor{FuncColor}{$\triangleright$\enspace\texttt{ReflexiveColors({\mdseries\slshape tensor})\index{ReflexiveColors@\texttt{ReflexiveColors}!for structure constants tensors}
\label{ReflexiveColors:for structure constants tensors}
}\hfill{\scriptsize (attribute)}}\\


 If \mbox{\texttt{\mdseries\slshape tensor}} is the structure constants tensor of the WL-stable color graph \texttt{cgr}, then \texttt{ReflexiveColors(tensor)} return the list of all reflexive colors of \texttt{cgr}. 
\begin{Verbatim}[commandchars=!@|,fontsize=\small,frame=single,label=Example]
  !gapprompt@gap>| !gapinput@e8:=Action(e8,AsList(e8), OnRight);|
  Group([ (1,2)(3,5)(4,6)(7,8), (1,3)(2,5)(4,7)(6,8), (1,4)(2,6)(3,7)(5,8) ])
  !gapprompt@gap>| !gapinput@cgr:=ColorGraph(e8,Combinations([1..DegreeAction(g)],2), OnSets);|
  <color graph of order 28 and rank 112>
  !gapprompt@gap>| !gapinput@T:=StructureConstantsOfColorGraph(cgr);|
  <Tensor of order 112>
  !gapprompt@gap>| !gapinput@ReflexiveColors(T);|
  [ 1, 18, 35, 52, 69, 86, 103 ]
  	    
\end{Verbatim}
 }

 

\subsection{\textcolor{Chapter }{NumberOfFibres (for structure constants tensors)}}
\logpage{[ 2, 3, 5 ]}\nobreak
\hyperdef{L}{X799D59167B7EA8FF}{}
{\noindent\textcolor{FuncColor}{$\triangleright$\enspace\texttt{NumberOfFibres({\mdseries\slshape tensor})\index{NumberOfFibres@\texttt{NumberOfFibres}!for structure constants tensors}
\label{NumberOfFibres:for structure constants tensors}
}\hfill{\scriptsize (attribute)}}\\


 Returns the number of reflexive colors of \mbox{\texttt{\mdseries\slshape tensor}}. }

 

\subsection{\textcolor{Chapter }{FibreLengths (for structure constants tensors)}}
\logpage{[ 2, 3, 6 ]}\nobreak
\hyperdef{L}{X82BDF90179606738}{}
{\noindent\textcolor{FuncColor}{$\triangleright$\enspace\texttt{FibreLengths({\mdseries\slshape tensor})\index{FibreLengths@\texttt{FibreLengths}!for structure constants tensors}
\label{FibreLengths:for structure constants tensors}
}\hfill{\scriptsize (attribute)}}\\


 If \mbox{\texttt{\mdseries\slshape tensor}} is the structure constants tensor of the WL-stable color graph \texttt{cgr}, then \texttt{FibreLengths(tensor)} returns the list of lengths of all fibres of \texttt{cgr}. The order corresponds to the result of \texttt{ReflexiveColors(T)}. 
\begin{Verbatim}[commandchars=!@|,fontsize=\small,frame=single,label=Example]
  !gapprompt@gap>| !gapinput@a5:=AlternatingGroup(5);|
  Alt( [ 1 .. 5 ] )
  !gapprompt@gap>| !gapinput@g:=Action(a5, Combinations([1..5],2), OnSets);|
  Group([ (1,5,8,10,4)(2,6,9,3,7), (2,3,4)(5,6,7)(8,10,9) ])
  !gapprompt@gap>| !gapinput@g:=Stabilizer(g,1);|
  Group([ (2,3,4)(5,6,7)(8,10,9), (2,6)(3,5)(4,7)(9,10) ])
  !gapprompt@gap>| !gapinput@cgr:=ColorGraph(g);|
  <color graph of order 10 and rank 19>
  !gapprompt@gap>| !gapinput@T:=StructureConstantsOfColorGraph(cgr);|
  <Tensor of order 19>
  !gapprompt@gap>| !gapinput@ReflexiveColors(T);|
  [ 1, 5, 18 ]
  !gapprompt@gap>| !gapinput@FibreLengths(T);|
  [ 1, 6, 3 ]
  !gapprompt@gap>| !gapinput@Fibres(cgr);|
  [ [ 1 ], [ 2, 3, 4, 5, 6, 7 ], [ 8, 9, 10 ] ]
  	    
\end{Verbatim}
 }

 

\subsection{\textcolor{Chapter }{OutValencies (for structure constants tensors)}}
\logpage{[ 2, 3, 7 ]}\nobreak
\hyperdef{L}{X7A59366582FCC6E7}{}
{\noindent\textcolor{FuncColor}{$\triangleright$\enspace\texttt{OutValencies({\mdseries\slshape tensor})\index{OutValencies@\texttt{OutValencies}!for structure constants tensors}
\label{OutValencies:for structure constants tensors}
}\hfill{\scriptsize (attribute)}}\\


 If \mbox{\texttt{\mdseries\slshape tensor}} is the structure constants tensor of the WL-stable color graph \texttt{cgr}, then \texttt{OutValencies(tensor)} returns the \texttt{OutValencies} (\ref{OutValencies:for WL-stable color graphs}) of \mbox{\texttt{\mdseries\slshape cgr}}. }

 

\subsection{\textcolor{Chapter }{Mates (for structure constants tensors)}}
\logpage{[ 2, 3, 8 ]}\nobreak
\hyperdef{L}{X7BEBEC40870FDC88}{}
{\noindent\textcolor{FuncColor}{$\triangleright$\enspace\texttt{Mates({\mdseries\slshape tensor})\index{Mates@\texttt{Mates}!for structure constants tensors}
\label{Mates:for structure constants tensors}
}\hfill{\scriptsize (attribute)}}\\


 If \mbox{\texttt{\mdseries\slshape tensor}} is the structure constants tensor of the WL-stable color graph \texttt{cgr}, then \texttt{OutValencies(tensor)} returns the permutation \texttt{ColorMates(cgr)} (\texttt{ColorMates} (\ref{ColorMates})). }

 

\subsection{\textcolor{Chapter }{StartBlock (for structure constants tensors)}}
\logpage{[ 2, 3, 9 ]}\nobreak
\hyperdef{L}{X7C07E78B7EF614F2}{}
{\noindent\textcolor{FuncColor}{$\triangleright$\enspace\texttt{StartBlock({\mdseries\slshape tensor, i})\index{StartBlock@\texttt{StartBlock}!for structure constants tensors}
\label{StartBlock:for structure constants tensors}
}\hfill{\scriptsize (operation)}}\\


 If \mbox{\texttt{\mdseries\slshape tensor}} is the structure constants tensor of the WL-stable color graph \texttt{cgr}, then in particular, the vertices of \mbox{\texttt{\mdseries\slshape tensor}} are the colors of \texttt{cgr}. All arcs of color \mbox{\texttt{\mdseries\slshape i}} have their starting vertex in the same fibre of \texttt{cgr}. Moreover, the loops over the vertices of one fibre all have the same color. 

 This function returns the index $j$ into \texttt{ReflexiveColors(T)} (cf. \texttt{ReflexiveColors} (\ref{ReflexiveColors:for structure constants tensors})) such that at the start of every arc of color \mbox{\texttt{\mdseries\slshape i}} there is a loop to color \texttt{ReflexiveColors(T)[j]}. }

 

\subsection{\textcolor{Chapter }{FinishBlock (for structure constants tensors)}}
\logpage{[ 2, 3, 10 ]}\nobreak
\hyperdef{L}{X87C78B90825DE43A}{}
{\noindent\textcolor{FuncColor}{$\triangleright$\enspace\texttt{FinishBlock({\mdseries\slshape tensor, i})\index{FinishBlock@\texttt{FinishBlock}!for structure constants tensors}
\label{FinishBlock:for structure constants tensors}
}\hfill{\scriptsize (operation)}}\\


 If \mbox{\texttt{\mdseries\slshape tensor}} is the structure constants tensor of the WL-stable color graph \texttt{cgr}, then in particular, the vertices of \mbox{\texttt{\mdseries\slshape tensor}} are the colors of \texttt{cgr}. All arcs of color \mbox{\texttt{\mdseries\slshape i}} have their finishing vertex in the same fibre of \texttt{cgr}. Moreover, the loops over the vertices of one fibre all have the same color. 

 This function returns the index $j$ into \texttt{ReflexiveColors(T)} (cf. \texttt{ReflexiveColors} (\ref{ReflexiveColors:for structure constants tensors})) such that at the end of every arc of color \mbox{\texttt{\mdseries\slshape i}} there is a loop to color \texttt{ReflexiveColors(T)[j]}. }

 

\subsection{\textcolor{Chapter }{ClosedSets}}
\logpage{[ 2, 3, 11 ]}\nobreak
\hyperdef{L}{X7C09E4D3805EFC78}{}
{\noindent\textcolor{FuncColor}{$\triangleright$\enspace\texttt{ClosedSets({\mdseries\slshape tensor})\index{ClosedSets@\texttt{ClosedSets}}
\label{ClosedSets}
}\hfill{\scriptsize (operation)}}\\


 A set $M$ of vertices of \mbox{\texttt{\mdseries\slshape tensor}} is called \emph{closed} if whenever $i,j$ are in $M$, then also all such $k$ are in $M$ for which \texttt{EntryOfTensor(tensor,i,j,k)} is non-zero. 

 This function returns all closed sets of \mbox{\texttt{\mdseries\slshape tensor}}. }

 

\subsection{\textcolor{Chapter }{ComplexProduct (for structure constants tensors)}}
\logpage{[ 2, 3, 12 ]}\nobreak
\hyperdef{L}{X79AB86E382D3F69D}{}
{\noindent\textcolor{FuncColor}{$\triangleright$\enspace\texttt{ComplexProduct({\mdseries\slshape tensor, set1, set2})\index{ComplexProduct@\texttt{ComplexProduct}!for structure constants tensors}
\label{ComplexProduct:for structure constants tensors}
}\hfill{\scriptsize (operation)}}\\


 Suppose that \mbox{\texttt{\mdseries\slshape tensor}} is the structure constants tensor of the WL-stale color graph \texttt{cgr}. the colors of \texttt{cgr} canonically correspond to the standard-basis elements of the coherent algebra $W$ that is associated with \mbox{\texttt{\mdseries\slshape cgr}}. The elements of $W$ can naturally be encoded as vectors of length \texttt{Rank(cgr)}. The arguments \mbox{\texttt{\mdseries\slshape set1}} and \mbox{\texttt{\mdseries\slshape set2}} are sets of colors of \texttt{cgr} (i.e. vertices of \mbox{\texttt{\mdseries\slshape tensor}}). Their characteristic vectors, can hence be understood as elements of $W$. 

 The operation \texttt{ComplexProduct} returns the coefficient-vector of the product of the characteristic vector of \mbox{\texttt{\mdseries\slshape set1}} with the characteristic vector of \mbox{\texttt{\mdseries\slshape set2}} in $W$. }

 

\subsection{\textcolor{Chapter }{ClosureSet}}
\logpage{[ 2, 3, 13 ]}\nobreak
\hyperdef{L}{X78C8C4CB8573D6D9}{}
{\noindent\textcolor{FuncColor}{$\triangleright$\enspace\texttt{ClosureSet({\mdseries\slshape tensor, set})\index{ClosureSet@\texttt{ClosureSet}}
\label{ClosureSet}
}\hfill{\scriptsize (function)}}\\


 \mbox{\texttt{\mdseries\slshape set}} is a set of vertices of \mbox{\texttt{\mdseries\slshape tensor}}. The function returns the smallest closed set of \mbox{\texttt{\mdseries\slshape tensor}} that contains \mbox{\texttt{\mdseries\slshape set}} (cf. \texttt{ClosedSets} (\ref{ClosedSets})) }

 }

 
\section{\textcolor{Chapter }{ Testing properties of tensors }}\logpage{[ 2, 4, 0 ]}
\hyperdef{L}{X81668D007D5A8982}{}
{
  

\subsection{\textcolor{Chapter }{IsTensorOfCC}}
\logpage{[ 2, 4, 1 ]}\nobreak
\hyperdef{L}{X7BD2B05E7CFF9F48}{}
{\noindent\textcolor{FuncColor}{$\triangleright$\enspace\texttt{IsTensorOfCC({\mdseries\slshape tensor})\index{IsTensorOfCC@\texttt{IsTensorOfCC}}
\label{IsTensorOfCC}
}\hfill{\scriptsize (property)}}\\


 If \mbox{\texttt{\mdseries\slshape tensor}} has this property, then this means that \textsf{COCO2P} knows, that it is the structure constants tensor of a WL-stable color graph.
There is no method installed for this property, as it is in general hard to
prove that a given tensor belongs to a WL-stable color graph. The property is
set by the constructor that created \mbox{\texttt{\mdseries\slshape tensor}}. }

 

\subsection{\textcolor{Chapter }{IsCommutativeTensor}}
\logpage{[ 2, 4, 2 ]}\nobreak
\hyperdef{L}{X7F9FD88986CBBA27}{}
{\noindent\textcolor{FuncColor}{$\triangleright$\enspace\texttt{IsCommutativeTensor({\mdseries\slshape tensor})\index{IsCommutativeTensor@\texttt{IsCommutativeTensor}}
\label{IsCommutativeTensor}
}\hfill{\scriptsize (property)}}\\


 \mbox{\texttt{\mdseries\slshape tensor}} has this property, if for all $i,j,k$ holds \texttt{EntryOfTensor(tensor,i,j,k)=} \texttt{EntryOfTensor(tensor,j,i,k)}. }

 

\subsection{\textcolor{Chapter }{IsHomogeneous (for structure constants tensors)}}
\logpage{[ 2, 4, 3 ]}\nobreak
\hyperdef{L}{X82E7A3EC7D036D9B}{}
{\noindent\textcolor{FuncColor}{$\triangleright$\enspace\texttt{IsHomogeneous({\mdseries\slshape tensor})\index{IsHomogeneous@\texttt{IsHomogeneous}!for structure constants tensors}
\label{IsHomogeneous:for structure constants tensors}
}\hfill{\scriptsize (property)}}\\


 Returns whether \mbox{\texttt{\mdseries\slshape tensor}} has just one reflexive color. }

 

\subsection{\textcolor{Chapter }{IsPrimitive (for structure constants tensors)}}
\logpage{[ 2, 4, 4 ]}\nobreak
\hyperdef{L}{X799D354A7A3C382D}{}
{\noindent\textcolor{FuncColor}{$\triangleright$\enspace\texttt{IsPrimitive({\mdseries\slshape tensor})\index{IsPrimitive@\texttt{IsPrimitive}!for structure constants tensors}
\label{IsPrimitive:for structure constants tensors}
}\hfill{\scriptsize (property)}}\\


 A structure constants tensor is \emph{primitive} if it is homogeneous and if it has only the trivial closed sets (i.e. the
singleton of the unique reflexive color and the set of all colors). 

 If \mbox{\texttt{\mdseries\slshape tensor}} is the structure constants tensor of the color graph \texttt{cgr}, then \mbox{\texttt{\mdseries\slshape tensor}} is primitive if and only if \texttt{cgr} is primitive (cf. \texttt{IsPrimitive} (\ref{IsPrimitive:for WL-stable color graphs})). }

 }

 
\section{\textcolor{Chapter }{Symmetries of tensors}}\logpage{[ 2, 5, 0 ]}
\hyperdef{L}{X83FB3E5587402E1D}{}
{
  

\subsection{\textcolor{Chapter }{KnownGroupOfAutomorphisms (for tensors)}}
\logpage{[ 2, 5, 1 ]}\nobreak
\hyperdef{L}{X8536FE8F84DA44E6}{}
{\noindent\textcolor{FuncColor}{$\triangleright$\enspace\texttt{KnownGroupOfAutomorphisms({\mdseries\slshape tensor})\index{KnownGroupOfAutomorphisms@\texttt{KnownGroupOfAutomorphisms}!for tensors}
\label{KnownGroupOfAutomorphisms:for tensors}
}\hfill{\scriptsize (operation)}}\\


 This function returns the group of all automorphisms of \mbox{\texttt{\mdseries\slshape tensor}} that \textsf{COCO2P} knows at the given moment. }

 

\subsection{\textcolor{Chapter }{AutGroupOfCocoObject (for tensors)}}
\logpage{[ 2, 5, 2 ]}\nobreak
\hyperdef{L}{X7A1AAC707A327410}{}
{\noindent\textcolor{FuncColor}{$\triangleright$\enspace\texttt{AutGroupOfCocoObject({\mdseries\slshape tensor})\index{AutGroupOfCocoObject@\texttt{AutGroupOfCocoObject}!for tensors}
\label{AutGroupOfCocoObject:for tensors}
}\hfill{\scriptsize (attribute)}}\\
\noindent\textcolor{FuncColor}{$\triangleright$\enspace\texttt{AutomorphismGroup({\mdseries\slshape tensor})\index{AutomorphismGroup@\texttt{AutomorphismGroup}!for tensors}
\label{AutomorphismGroup:for tensors}
}\hfill{\scriptsize (method)}}\\


 Returns the group of all automorphisms of \mbox{\texttt{\mdseries\slshape tensor}}. }

 

\subsection{\textcolor{Chapter }{IsAutomorphismOfObject (for tensors)}}
\logpage{[ 2, 5, 3 ]}\nobreak
\hyperdef{L}{X7B5F5D7A7DA27A8D}{}
{\noindent\textcolor{FuncColor}{$\triangleright$\enspace\texttt{IsAutomorphismOfObject({\mdseries\slshape tensor, perm})\index{IsAutomorphismOfObject@\texttt{IsAutomorphismOfObject}!for tensors}
\label{IsAutomorphismOfObject:for tensors}
}\hfill{\scriptsize (operation)}}\\
\noindent\textcolor{FuncColor}{$\triangleright$\enspace\texttt{IsAutomorphismOfTensor({\mdseries\slshape tensor, perm})\index{IsAutomorphismOfTensor@\texttt{IsAutomorphismOfTensor}!for tensors}
\label{IsAutomorphismOfTensor:for tensors}
}\hfill{\scriptsize (operation)}}\\


 Returns \texttt{true}, if \mbox{\texttt{\mdseries\slshape perm}} is an automorphism of \mbox{\texttt{\mdseries\slshape tensor}}. In that case \textsf{COCO2P} adds \mbox{\texttt{\mdseries\slshape perm}} to the known automorphisms of \mbox{\texttt{\mdseries\slshape tensor}}. }

 

\subsection{\textcolor{Chapter }{IsomorphismCocoObjects (for tensors)}}
\logpage{[ 2, 5, 4 ]}\nobreak
\hyperdef{L}{X7EE1D367802B6726}{}
{\noindent\textcolor{FuncColor}{$\triangleright$\enspace\texttt{IsomorphismCocoObjects({\mdseries\slshape tensor1, tensor2})\index{IsomorphismCocoObjects@\texttt{IsomorphismCocoObjects}!for tensors}
\label{IsomorphismCocoObjects:for tensors}
}\hfill{\scriptsize (operation)}}\\
\noindent\textcolor{FuncColor}{$\triangleright$\enspace\texttt{IsomorphismTensors({\mdseries\slshape tensor1, tensor2})\index{IsomorphismTensors@\texttt{IsomorphismTensors}!for tensors}
\label{IsomorphismTensors:for tensors}
}\hfill{\scriptsize (operation)}}\\


 This operation returns an isomorphism from \mbox{\texttt{\mdseries\slshape tensor1}} to \mbox{\texttt{\mdseries\slshape tensor2}} if it exists, and \texttt{fail} if it does not exists. }

 

\subsection{\textcolor{Chapter }{IsIsomorphicCocoObject (for tensors)}}
\logpage{[ 2, 5, 5 ]}\nobreak
\hyperdef{L}{X867CF32F827578F8}{}
{\noindent\textcolor{FuncColor}{$\triangleright$\enspace\texttt{IsIsomorphicCocoObject({\mdseries\slshape tensor1, tensor2})\index{IsIsomorphicCocoObject@\texttt{IsIsomorphicCocoObject}!for tensors}
\label{IsIsomorphicCocoObject:for tensors}
}\hfill{\scriptsize (operation)}}\\
\noindent\textcolor{FuncColor}{$\triangleright$\enspace\texttt{IsIsomorphicTensor({\mdseries\slshape tensor1, tensor2})\index{IsIsomorphicTensor@\texttt{IsIsomorphicTensor}!for tensors}
\label{IsIsomorphicTensor:for tensors}
}\hfill{\scriptsize (operation)}}\\


 Returns \texttt{true} if \mbox{\texttt{\mdseries\slshape tensor1}} and \mbox{\texttt{\mdseries\slshape tensor2}} are isomorphic, and \texttt{false} otherwise. }

 

\subsection{\textcolor{Chapter }{IsIsomorphismOfObjects (for tensors)}}
\logpage{[ 2, 5, 6 ]}\nobreak
\hyperdef{L}{X84122BB6867A77CB}{}
{\noindent\textcolor{FuncColor}{$\triangleright$\enspace\texttt{IsIsomorphismOfObjects({\mdseries\slshape tensor1, tensor2, g})\index{IsIsomorphismOfObjects@\texttt{IsIsomorphismOfObjects}!for tensors}
\label{IsIsomorphismOfObjects:for tensors}
}\hfill{\scriptsize (operation)}}\\
\noindent\textcolor{FuncColor}{$\triangleright$\enspace\texttt{IsIsomorphismOfTensors({\mdseries\slshape tensor1, tensor2, g})\index{IsIsomorphismOfTensors@\texttt{IsIsomorphismOfTensors}!for tensors}
\label{IsIsomorphismOfTensors:for tensors}
}\hfill{\scriptsize (operation)}}\\


 Returns \texttt{true} if \mbox{\texttt{\mdseries\slshape g}} is an isomorphism fro \mbox{\texttt{\mdseries\slshape tensor1}} to \mbox{\texttt{\mdseries\slshape tensor2}}, and \texttt{false} otherwise. }

 }

 
\section{\textcolor{Chapter }{Character tables of structure constants tensors}}\logpage{[ 2, 6, 0 ]}
\hyperdef{L}{X7AC08470842F62FB}{}
{
  The structure constants tensor of a WL-stable color graph encodes the
structure of the associated coherent algebra. If this algebra is commutative,
then \textsf{COCO2P} is able to compute its character table provided, the irrationalities occuring
are representable in \textsf{GAP}. The algorithm that computes the character tables involves Gr{\"o}bner-bases.
The computation of the Gr{\"o}bner bases defines the overall performance of
the algorithm for the computation of character tables. 

\subsection{\textcolor{Chapter }{CharacterTableOfTensor (for commutative structure constants tensors)}}
\logpage{[ 2, 6, 1 ]}\nobreak
\hyperdef{L}{X7B8C349685845950}{}
{\noindent\textcolor{FuncColor}{$\triangleright$\enspace\texttt{CharacterTableOfTensor({\mdseries\slshape tensor})\index{CharacterTableOfTensor@\texttt{CharacterTableOfTensor}!for commutative structure constants tensors}
\label{CharacterTableOfTensor:for commutative structure constants tensors}
}\hfill{\scriptsize (attribute)}}\\


 This function returns a record with two components: \texttt{characters} and \texttt{multiplicities}. If \texttt{ct} is the character table of \mbox{\texttt{\mdseries\slshape tensor}}, then \texttt{ct.characters[i][j]} is the value of the $i$-th irreducible character of the standard-basis element corresponding to color $j$ of \mbox{\texttt{\mdseries\slshape tensor}}. Moreover, \texttt{ct.multiplicities[i]} is the multiplicity of the $i$-th irreducible character. 
\begin{Verbatim}[commandchars=!@|,fontsize=\small,frame=single,label=Example]
  !gapprompt@gap>| !gapinput@cgr:=JohnsonScheme(6,3);|
  <color graph of order 20 and rank 4>
  !gapprompt@gap>| !gapinput@T:=StructureConstantsOfColorGraph(cgr);|
  <Tensor of order 4>
  !gapprompt@gap>| !gapinput@IsCommutativeTensor(T);|
  true
  !gapprompt@gap>| !gapinput@CharacterTable(T);|
  rec( characters := [ [ 1, 9, 9, 1 ], [ 1, -1, -1, 1 ], [ 1, -3, 3, -1 ],
        [ 1, 3, -3, -1 ] ], multiplicities := [ 1, 9, 5, 5 ] )
  	    
\end{Verbatim}
 }

 }

 }

 
\chapter{\textcolor{Chapter }{WL-Stable Fusions Color Graphs}}\logpage{[ 3, 0, 0 ]}
\hyperdef{L}{X81EC7ADC7F919C22}{}
{
  
\section{\textcolor{Chapter }{Introduction}}\logpage{[ 3, 1, 0 ]}
\hyperdef{L}{X7DFB63A97E67C0A1}{}
{
  One of the fundamental methods how to derive new color graphs from a color
graph $\Gamma$, is to \emph{fuse} (i.e identify) colors. Color graphs that are derived from $\Gamma$ in this way are called \emph{fusion color graphs}. Every fusion color graph $\Delta$ of $\Gamma$ defines a partition on the colors of $\Gamma$. This partition is called the \emph{fusion} associated with the fusion color graph $\Delta$ of $\Gamma$. If $\Delta$ is WL-stable, then its fusion is called a \emph{stable fusion}. 

 One of the fundamental algorithmical problems in algebraic combinatorics is to
enumerate all WL-stable fusion color graphs of a given color graph. At the
moment \textsf{COCO2P} can solve a part of this problem {\textendash} namely starting from any
WL-stable color graph $\Gamma$ it can enumerate (orbits of) stable fusions that lead to homogeneous WL-stable
fusion color graphs. Such fusions we will call \emph{homogeneous}. 

 Computing stable fusions, in \textsf{COCO2P} is a two-stages process: 
\begin{enumerate}
\item Computation of good sets of colors,
\item Fitting together good sets to stable fusions.
\end{enumerate}
 Good sets are the building blocks of stable fusions. A set of colors of a
WL-stable color graph is called a \emph{good set} if there exists a stable fusion of the cgr in which the set appears as a
class. It is called a \emph{homogeneous good set} if it is part of a homogeneous stable fusion. Note that the property to be a
(homogeneous) good set does only depend on the structure constants of the
color graph. }

 
\section{\textcolor{Chapter }{Good sets}}\logpage{[ 3, 2, 0 ]}
\hyperdef{L}{X8615A23A878B185D}{}
{
  

\subsection{\textcolor{Chapter }{BuildGoodSet}}
\logpage{[ 3, 2, 1 ]}\nobreak
\hyperdef{L}{X7D8BBE457812993B}{}
{\noindent\textcolor{FuncColor}{$\triangleright$\enspace\texttt{BuildGoodSet({\mdseries\slshape tensor, set[, part]})\index{BuildGoodSet@\texttt{BuildGoodSet}}
\label{BuildGoodSet}
}\hfill{\scriptsize (function)}}\\


 \mbox{\texttt{\mdseries\slshape tensor}} is the structure constants tensor of a WL-stable color graph \texttt{cgr}. \mbox{\texttt{\mdseries\slshape set}} is a set of colors of \texttt{cgr} (i.e. of vertices of \mbox{\texttt{\mdseries\slshape tensor}}). \mbox{\texttt{\mdseries\slshape part}} is supposed to be the coarsest stable partition of the colors of \texttt{cgr} that contains \mbox{\texttt{\mdseries\slshape set}} as a class (the stability is not checked by the function). The function
returns the corresponding good-set object. 

 If \mbox{\texttt{\mdseries\slshape part}} is not given, then it is computed. If this computation fails (because \mbox{\texttt{\mdseries\slshape set}} is not a good set), then \texttt{fail} is returned. }

 

\subsection{\textcolor{Chapter }{AsSet (for good sets)}}
\logpage{[ 3, 2, 2 ]}\nobreak
\hyperdef{L}{X7DC2B6E27A050E8F}{}
{\noindent\textcolor{FuncColor}{$\triangleright$\enspace\texttt{AsSet({\mdseries\slshape gs})\index{AsSet@\texttt{AsSet}!for good sets}
\label{AsSet:for good sets}
}\hfill{\scriptsize (operation)}}\\


 Converts the good set object \mbox{\texttt{\mdseries\slshape gs}} into a usual set. }

 

\subsection{\textcolor{Chapter }{TensorOfGoodSet}}
\logpage{[ 3, 2, 3 ]}\nobreak
\hyperdef{L}{X7B76EDA4800DAB0F}{}
{\noindent\textcolor{FuncColor}{$\triangleright$\enspace\texttt{TensorOfGoodSet({\mdseries\slshape gs})\index{TensorOfGoodSet@\texttt{TensorOfGoodSet}}
\label{TensorOfGoodSet}
}\hfill{\scriptsize (operation)}}\\


 Returns the structure constants tensor over which the good set \mbox{\texttt{\mdseries\slshape gs}} is ``good''. }

 

\subsection{\textcolor{Chapter }{PartitionOfGoodSet}}
\logpage{[ 3, 2, 4 ]}\nobreak
\hyperdef{L}{X866BB45B7AE2320B}{}
{\noindent\textcolor{FuncColor}{$\triangleright$\enspace\texttt{PartitionOfGoodSet({\mdseries\slshape gs})\index{PartitionOfGoodSet@\texttt{PartitionOfGoodSet}}
\label{PartitionOfGoodSet}
}\hfill{\scriptsize (operation)}}\\


 This function returns the coarsest stable fusion (as a partition, i.e. a set
of sets of colors), that contains \mbox{\texttt{\mdseries\slshape gs}} as a class. }

 }

 
\section{\textcolor{Chapter }{Orbits of good sets}}\logpage{[ 3, 3, 0 ]}
\hyperdef{L}{X7C8FC58D8552CAB2}{}
{
  \textsf{COCO2P} implements a datatype for orbits of combinatorial objects. This section
describes the functions that deal with orbits of good sets. For every orbit of
good sets, only the lexicographically smallest representative and its set-wise
stabilizer is saved. This allows to deal with good sets of color graphs of
comparatively high rank, provided they have many algebraic automorphisms. 

\subsection{\textcolor{Chapter }{HomogeneousGoodSetOrbits (for structure constants tensors)}}
\logpage{[ 3, 3, 1 ]}\nobreak
\hyperdef{L}{X86ACF5D278B008B0}{}
{\noindent\textcolor{FuncColor}{$\triangleright$\enspace\texttt{HomogeneousGoodSetOrbits({\mdseries\slshape tensor})\index{HomogeneousGoodSetOrbits@\texttt{HomogeneousGoodSetOrbits}!for structure constants tensors}
\label{HomogeneousGoodSetOrbits:for structure constants tensors}
}\hfill{\scriptsize (attribute)}}\\
\noindent\textcolor{FuncColor}{$\triangleright$\enspace\texttt{HomogeneousGoodSetOrbits({\mdseries\slshape group, tensor[, mode]})\index{HomogeneousGoodSetOrbits@\texttt{HomogeneousGoodSetOrbits}!for structure constants tensors, alternative}
\label{HomogeneousGoodSetOrbits:for structure constants tensors, alternative}
}\hfill{\scriptsize (method)}}\\


 \mbox{\texttt{\mdseries\slshape group}} is supposed to consist only of automorphisms of \mbox{\texttt{\mdseries\slshape tensor}}. \textsf{COCO2P} learns new automorphisms by checking this property. If group is not given,
then the full automorphism group of \mbox{\texttt{\mdseries\slshape tensor}} is taken for \mbox{\texttt{\mdseries\slshape group}}. 

 This function returns all \mbox{\texttt{\mdseries\slshape group}}-orbits of homogeneous good sets. 

 If \mbox{\texttt{\mdseries\slshape mode}} is equal to \texttt{"s"}, then only orbits of symmetric good sets are returned. If \mbox{\texttt{\mdseries\slshape mode}} is equal to \texttt{"a"}, then only orbits of asymmetric good sets are returned. }

 

\subsection{\textcolor{Chapter }{GoodSetOrbit}}
\logpage{[ 3, 3, 2 ]}\nobreak
\hyperdef{L}{X7F6548818672AA8F}{}
{\noindent\textcolor{FuncColor}{$\triangleright$\enspace\texttt{GoodSetOrbit({\mdseries\slshape group, gs[, stab]})\index{GoodSetOrbit@\texttt{GoodSetOrbit}}
\label{GoodSetOrbit}
}\hfill{\scriptsize (operation)}}\\


 \mbox{\texttt{\mdseries\slshape gs}} is a good set. \mbox{\texttt{\mdseries\slshape group}} has to be a subgroup of the automorphism group of \texttt{TensorOfGoodSet(gs)}. \mbox{\texttt{\mdseries\slshape stab}} (if given) has to be the full set-wise stabilizer of \mbox{\texttt{\mdseries\slshape gs}} in \mbox{\texttt{\mdseries\slshape group}}. 

 The function constructs a \textsf{COCO2P}-orbit object of the setwise orbit of \mbox{\texttt{\mdseries\slshape gs}} under \mbox{\texttt{\mdseries\slshape group}}. }

 

\subsection{\textcolor{Chapter }{CanonicalRepresentativeOfCocoOrbit (for orbits of good sets)}}
\logpage{[ 3, 3, 3 ]}\nobreak
\hyperdef{L}{X86574C0C8000226E}{}
{\noindent\textcolor{FuncColor}{$\triangleright$\enspace\texttt{CanonicalRepresentativeOfCocoOrbit({\mdseries\slshape gsorb})\index{CanonicalRepresentativeOfCocoOrbit@\texttt{CanonicalRepresentativeOfCocoOrbit}!for orbits of good sets}
\label{CanonicalRepresentativeOfCocoOrbit:for orbits of good sets}
}\hfill{\scriptsize (operation)}}\\


 This function returns the lexicographically smallest element of the orbit of
good sets \mbox{\texttt{\mdseries\slshape gsorb}}. }

 

\subsection{\textcolor{Chapter }{Representative (for orbits of good sets)}}
\logpage{[ 3, 3, 4 ]}\nobreak
\hyperdef{L}{X7ADE295E7CB04AF7}{}
{\noindent\textcolor{FuncColor}{$\triangleright$\enspace\texttt{Representative({\mdseries\slshape gsorb})\index{Representative@\texttt{Representative}!for orbits of good sets}
\label{Representative:for orbits of good sets}
}\hfill{\scriptsize (operation)}}\\


 This function returns any element of the orbit of good sets \mbox{\texttt{\mdseries\slshape gsorb}}. At the moment it in fact returns the lexicographically smallest element. }

 

\subsection{\textcolor{Chapter }{UnderlyingGroupOfCocoOrbit (for orbits of good sets)}}
\logpage{[ 3, 3, 5 ]}\nobreak
\hyperdef{L}{X7B8CEEE08351083F}{}
{\noindent\textcolor{FuncColor}{$\triangleright$\enspace\texttt{UnderlyingGroupOfCocoOrbit({\mdseries\slshape gsorb})\index{UnderlyingGroupOfCocoOrbit@\texttt{UnderlyingGroupOfCocoOrbit}!for orbits of good sets}
\label{UnderlyingGroupOfCocoOrbit:for orbits of good sets}
}\hfill{\scriptsize (operation)}}\\


 This function returns the group under which \mbox{\texttt{\mdseries\slshape gsorb}} is an orbit. }

 

\subsection{\textcolor{Chapter }{StabilizerOfCanonicalRepresentative (for orbits of good sets)}}
\logpage{[ 3, 3, 6 ]}\nobreak
\hyperdef{L}{X7EACF062833AF3FF}{}
{\noindent\textcolor{FuncColor}{$\triangleright$\enspace\texttt{StabilizerOfCanonicalRepresentative({\mdseries\slshape gsorb})\index{StabilizerOfCanonicalRepresentative@\texttt{StabilizerOfCanonicalRepresentative}!for orbits of good sets}
\label{StabilizerOfCanonicalRepresentative:for orbits of good sets}
}\hfill{\scriptsize (operation)}}\\


 This function returns the setwise stabilizer of \texttt{CanonicalRepresentativeOfCocoOrbit(gsorb)} in \texttt{UnderlyingGroupOfCocoOrbit(gsorb)}. }

 

\subsection{\textcolor{Chapter }{Size (for orbits of good sets)}}
\logpage{[ 3, 3, 7 ]}\nobreak
\hyperdef{L}{X79AE3E387AB90F45}{}
{\noindent\textcolor{FuncColor}{$\triangleright$\enspace\texttt{Size({\mdseries\slshape gsorb})\index{Size@\texttt{Size}!for orbits of good sets}
\label{Size:for orbits of good sets}
}\hfill{\scriptsize (method)}}\\


 returns the size of \mbox{\texttt{\mdseries\slshape gsorb}}. }

 

\subsection{\textcolor{Chapter }{AsList (for orbits of good sets)}}
\logpage{[ 3, 3, 8 ]}\nobreak
\hyperdef{L}{X8334694B7EC2CEC3}{}
{\noindent\textcolor{FuncColor}{$\triangleright$\enspace\texttt{AsList({\mdseries\slshape gsorb})\index{AsList@\texttt{AsList}!for orbits of good sets}
\label{AsList:for orbits of good sets}
}\hfill{\scriptsize (method)}}\\


 expands the \textsf{COCO2P}-orbit object \mbox{\texttt{\mdseries\slshape gsorb}} into a list of good sets. }

 

\subsection{\textcolor{Chapter }{AsSet (for orbits of good sets)}}
\logpage{[ 3, 3, 9 ]}\nobreak
\hyperdef{L}{X7CFBE61D8153E4F9}{}
{\noindent\textcolor{FuncColor}{$\triangleright$\enspace\texttt{AsSet({\mdseries\slshape gsorb})\index{AsSet@\texttt{AsSet}!for orbits of good sets}
\label{AsSet:for orbits of good sets}
}\hfill{\scriptsize (method)}}\\


 expands the \textsf{COCO2P}-orbit object \mbox{\texttt{\mdseries\slshape gsorb}} into a set of good sets. }

 

\subsection{\textcolor{Chapter }{SubOrbitsOfCocoOrbit (for orbits of good sets)}}
\logpage{[ 3, 3, 10 ]}\nobreak
\hyperdef{L}{X85280FDF86123458}{}
{\noindent\textcolor{FuncColor}{$\triangleright$\enspace\texttt{SubOrbitsOfCocoOrbit({\mdseries\slshape group, gsorb})\index{SubOrbitsOfCocoOrbit@\texttt{SubOrbitsOfCocoOrbit}!for orbits of good sets}
\label{SubOrbitsOfCocoOrbit:for orbits of good sets}
}\hfill{\scriptsize (operation)}}\\


 \mbox{\texttt{\mdseries\slshape group}} is a subgroup of the underlying group of the orbit of good sets \mbox{\texttt{\mdseries\slshape gsorb}}. The given orbit splits into suborbits under this group. The function returns
a list of these suborbits. }

 

\subsection{\textcolor{Chapter }{SubOrbitsWithInvariantPropertyOfCocoOrbit (for orbits of good sets)}}
\logpage{[ 3, 3, 11 ]}\nobreak
\hyperdef{L}{X79370B047C27512A}{}
{\noindent\textcolor{FuncColor}{$\triangleright$\enspace\texttt{SubOrbitsWithInvariantPropertyOfCocoOrbit({\mdseries\slshape group, gsorb, prop})\index{SubOrbitsWithInvariantPropertyOfCocoOrbit@\texttt{Sub}\-\texttt{Orbits}\-\texttt{With}\-\texttt{Invariant}\-\texttt{Property}\-\texttt{Of}\-\texttt{Coco}\-\texttt{Orbit}!for orbits of good sets}
\label{SubOrbitsWithInvariantPropertyOfCocoOrbit:for orbits of good sets}
}\hfill{\scriptsize (operation)}}\\


 \mbox{\texttt{\mdseries\slshape prop}} is a function that takes a single good set as argument and returns \texttt{true} or \texttt{false}. It has to be invariant under the set-wise action of \mbox{\texttt{\mdseries\slshape group}}. Note that this property is not checked by the function. 

 This function does the same as 
\begin{Verbatim}[fontsize=\small,frame=single,label=]
  Filtered(SubOrbitsOfCocoOrbit(group,gsorb), x->prop(Representative(x)));
\end{Verbatim}
 However, the former code is generally much less efficient than calling 
\begin{Verbatim}[fontsize=\small,frame=single,label=]
  SubOrbitsWithInvariantPropertyOfCocoOrbit(group,gsorb,prop);
\end{Verbatim}
 }

 }

 
\section{\textcolor{Chapter }{Fusions}}\logpage{[ 3, 4, 0 ]}
\hyperdef{L}{X791CDD977B9FD97A}{}
{
  

\subsection{\textcolor{Chapter }{FusionFromPartition (for structure constant tensors)}}
\logpage{[ 3, 4, 1 ]}\nobreak
\hyperdef{L}{X87B7F62F7C5CDCAE}{}
{\noindent\textcolor{FuncColor}{$\triangleright$\enspace\texttt{FusionFromPartition({\mdseries\slshape tensor, part})\index{FusionFromPartition@\texttt{FusionFromPartition}!for structure constant tensors}
\label{FusionFromPartition:for structure constant tensors}
}\hfill{\scriptsize (function)}}\\


 if \mbox{\texttt{\mdseries\slshape tensor}} is the structure constants tensor of the WL-stable color graph \texttt{cgr}, and if \mbox{\texttt{\mdseries\slshape part}} is a partition of the colors of \texttt{cgr} (a set of sets of colors), then this function returns a fusion-object, or \texttt{fail} if \mbox{\texttt{\mdseries\slshape part}} is not a fusion of \texttt{cgr}. }

 

\subsection{\textcolor{Chapter }{AsPartition}}
\logpage{[ 3, 4, 2 ]}\nobreak
\hyperdef{L}{X8493C9117A37FDEE}{}
{\noindent\textcolor{FuncColor}{$\triangleright$\enspace\texttt{AsPartition({\mdseries\slshape fusion})\index{AsPartition@\texttt{AsPartition}}
\label{AsPartition}
}\hfill{\scriptsize (attribute)}}\\


 Converts the fusion-object \mbox{\texttt{\mdseries\slshape fusion}} into a set of sets of colors. }

 

\subsection{\textcolor{Chapter }{PartitionOfFusion}}
\logpage{[ 3, 4, 3 ]}\nobreak
\hyperdef{L}{X83FC0D2881F64830}{}
{\noindent\textcolor{FuncColor}{$\triangleright$\enspace\texttt{PartitionOfFusion({\mdseries\slshape fusion})\index{PartitionOfFusion@\texttt{PartitionOfFusion}}
\label{PartitionOfFusion}
}\hfill{\scriptsize (operation)}}\\


 Converts the fusion object \mbox{\texttt{\mdseries\slshape fusion}} into a list of sets. In contrast to te result of \texttt{AsPartition(fusion)}, the resulting list of classes is sorted in short-lex order. This means that
first it is sorted by cardinality of classes, and then the classes of equal
size are sorted lexicographically. }

 

\subsection{\textcolor{Chapter }{TensorOfFusion}}
\logpage{[ 3, 4, 4 ]}\nobreak
\hyperdef{L}{X85138D32850D82E0}{}
{\noindent\textcolor{FuncColor}{$\triangleright$\enspace\texttt{TensorOfFusion({\mdseries\slshape fusion})\index{TensorOfFusion@\texttt{TensorOfFusion}}
\label{TensorOfFusion}
}\hfill{\scriptsize (operation)}}\\


 returns the structure constants tensor, over which the fusion \mbox{\texttt{\mdseries\slshape fusion}} is a stable fusion. }

 

\subsection{\textcolor{Chapter }{BaseOfFusion}}
\logpage{[ 3, 4, 5 ]}\nobreak
\hyperdef{L}{X8770B6517FBFCC4C}{}
{\noindent\textcolor{FuncColor}{$\triangleright$\enspace\texttt{BaseOfFusion({\mdseries\slshape fusion})\index{BaseOfFusion@\texttt{BaseOfFusion}}
\label{BaseOfFusion}
}\hfill{\scriptsize (attribute)}}\\


 The base of a fusion is the smallest list of classes of \mbox{\texttt{\mdseries\slshape fusion}} (in the short lex order) that generates \mbox{\texttt{\mdseries\slshape fusion}} in the sense that there is no coarser stable fusion that contains the classes
of the base. 

 This function returns the base of \mbox{\texttt{\mdseries\slshape fusion}} if \textsf{COCO2P} knows it. At the moment there is no method for computing the base. }

 

\subsection{\textcolor{Chapter }{RankOfFusion}}
\logpage{[ 3, 4, 6 ]}\nobreak
\hyperdef{L}{X861815928126A93B}{}
{\noindent\textcolor{FuncColor}{$\triangleright$\enspace\texttt{RankOfFusion({\mdseries\slshape fusion})\index{RankOfFusion@\texttt{RankOfFusion}}
\label{RankOfFusion}
}\hfill{\scriptsize (attribute)}}\\


 returns the number of classes of \mbox{\texttt{\mdseries\slshape fusion}}. }

 }

 
\section{\textcolor{Chapter }{ Orbits of fusions}}\logpage{[ 3, 5, 0 ]}
\hyperdef{L}{X79E3348682BC2376}{}
{
  \textsf{COCO2P} implements a datatype for orbits of combinatorial objects. This section
describes the functions that deal with orbits of stable fusion. For every
orbit of fusions, only the smallest representative in the short-lex order and
its partition-wise stabilizer is saved. This allows to deal with fusions of
color graphs of comparatively high rank. 

\subsection{\textcolor{Chapter }{HomogeneousFusionOrbits (for structure constants tensors)}}
\logpage{[ 3, 5, 1 ]}\nobreak
\hyperdef{L}{X7D1B9BF879F637D7}{}
{\noindent\textcolor{FuncColor}{$\triangleright$\enspace\texttt{HomogeneousFusionOrbits({\mdseries\slshape tensor})\index{HomogeneousFusionOrbits@\texttt{HomogeneousFusionOrbits}!for structure constants tensors}
\label{HomogeneousFusionOrbits:for structure constants tensors}
}\hfill{\scriptsize (attribute)}}\\
\noindent\textcolor{FuncColor}{$\triangleright$\enspace\texttt{HomogeneousFusionOrbits({\mdseries\slshape group, tensor})\index{HomogeneousFusionOrbits@\texttt{HomogeneousFusionOrbits}!for structure constants tensors, alternative}
\label{HomogeneousFusionOrbits:for structure constants tensors, alternative}
}\hfill{\scriptsize (method)}}\\


 \mbox{\texttt{\mdseries\slshape group}} is supposed to consist only of automorphisms of \mbox{\texttt{\mdseries\slshape tensor}}. \textsf{COCO2P} learns new automorphisms by checking this property. If group is not given,
then the full automorphism group of \mbox{\texttt{\mdseries\slshape tensor}} is taken for \mbox{\texttt{\mdseries\slshape group}}. 

 This function returns all \mbox{\texttt{\mdseries\slshape group}}-orbits of homogeneous stable fusions. }

 

\subsection{\textcolor{Chapter }{FusionOrbit}}
\logpage{[ 3, 5, 2 ]}\nobreak
\hyperdef{L}{X82E731437F02346D}{}
{\noindent\textcolor{FuncColor}{$\triangleright$\enspace\texttt{FusionOrbit({\mdseries\slshape group, fusion[, stab]})\index{FusionOrbit@\texttt{FusionOrbit}}
\label{FusionOrbit}
}\hfill{\scriptsize (operation)}}\\


 \mbox{\texttt{\mdseries\slshape fusion}} is a fusion object. \mbox{\texttt{\mdseries\slshape group}} has to be a subgroup of the automorphism group of \texttt{TensorOfFusion(fusion)}. \mbox{\texttt{\mdseries\slshape stab}} (if given) has to be the full partition-wise stabilizer of \mbox{\texttt{\mdseries\slshape fusion}} in \mbox{\texttt{\mdseries\slshape group}}. 

 The function constructs a \textsf{COCO2P}-orbit object of the partition-wise orbit of \mbox{\texttt{\mdseries\slshape fusion}} under \mbox{\texttt{\mdseries\slshape group}}. }

 

\subsection{\textcolor{Chapter }{CanonicalRepresentativeOfCocoOrbit (for orbits of fusions)}}
\logpage{[ 3, 5, 3 ]}\nobreak
\hyperdef{L}{X85B7C08285E5B799}{}
{\noindent\textcolor{FuncColor}{$\triangleright$\enspace\texttt{CanonicalRepresentativeOfCocoOrbit({\mdseries\slshape fusionorb})\index{CanonicalRepresentativeOfCocoOrbit@\texttt{CanonicalRepresentativeOfCocoOrbit}!for orbits of fusions}
\label{CanonicalRepresentativeOfCocoOrbit:for orbits of fusions}
}\hfill{\scriptsize (operation)}}\\


 This function returns the smallest element (in the short-lex order) of the
orbit of fusions \mbox{\texttt{\mdseries\slshape fusionorb}}. }

 

\subsection{\textcolor{Chapter }{Representative (for orbits of fusions)}}
\logpage{[ 3, 5, 4 ]}\nobreak
\hyperdef{L}{X85BAA5A87BA76215}{}
{\noindent\textcolor{FuncColor}{$\triangleright$\enspace\texttt{Representative({\mdseries\slshape fusionorb})\index{Representative@\texttt{Representative}!for orbits of fusions}
\label{Representative:for orbits of fusions}
}\hfill{\scriptsize (operation)}}\\


 This function returns any element of the orbit of fusions sets \mbox{\texttt{\mdseries\slshape fusionorb}}. At the moment it in fact returns the canonical representative. }

 

\subsection{\textcolor{Chapter }{UnderlyingGroupOfCocoOrbit (for orbits of fusions)}}
\logpage{[ 3, 5, 5 ]}\nobreak
\hyperdef{L}{X7B4255C57867788C}{}
{\noindent\textcolor{FuncColor}{$\triangleright$\enspace\texttt{UnderlyingGroupOfCocoOrbit({\mdseries\slshape fusionorb})\index{UnderlyingGroupOfCocoOrbit@\texttt{UnderlyingGroupOfCocoOrbit}!for orbits of fusions}
\label{UnderlyingGroupOfCocoOrbit:for orbits of fusions}
}\hfill{\scriptsize (operation)}}\\


 This function returns the group under which \mbox{\texttt{\mdseries\slshape fusionorb}} is an orbit. }

 

\subsection{\textcolor{Chapter }{StabilizerOfCanonicalRepresentative (for orbits of fusions)}}
\logpage{[ 3, 5, 6 ]}\nobreak
\hyperdef{L}{X7B8F411C863F4179}{}
{\noindent\textcolor{FuncColor}{$\triangleright$\enspace\texttt{StabilizerOfCanonicalRepresentative({\mdseries\slshape fusion})\index{StabilizerOfCanonicalRepresentative@\texttt{StabilizerOfCanonicalRepresentative}!for orbits of fusions}
\label{StabilizerOfCanonicalRepresentative:for orbits of fusions}
}\hfill{\scriptsize (operation)}}\\


 This function returns the partition-wise stabilizer of \texttt{CanonicalRepresentativeOfCocoOrbit(fusionorb)} in \texttt{UnderlyingGroupOfCocoOrbit(fusionorb)}. }

 

\subsection{\textcolor{Chapter }{Size (for orbits of fusions)}}
\logpage{[ 3, 5, 7 ]}\nobreak
\hyperdef{L}{X7C0606D6878E1F35}{}
{\noindent\textcolor{FuncColor}{$\triangleright$\enspace\texttt{Size({\mdseries\slshape fusionorb})\index{Size@\texttt{Size}!for orbits of fusions}
\label{Size:for orbits of fusions}
}\hfill{\scriptsize (method)}}\\


 returns the size of \mbox{\texttt{\mdseries\slshape fusionorb}}. }

 

\subsection{\textcolor{Chapter }{AsList (for orbits of fusions)}}
\logpage{[ 3, 5, 8 ]}\nobreak
\hyperdef{L}{X87538F0F86B539BA}{}
{\noindent\textcolor{FuncColor}{$\triangleright$\enspace\texttt{AsList({\mdseries\slshape fusionorb})\index{AsList@\texttt{AsList}!for orbits of fusions}
\label{AsList:for orbits of fusions}
}\hfill{\scriptsize (method)}}\\


s expands the \textsf{COCO2P}-orbit object \mbox{\texttt{\mdseries\slshape fusionorb}} into a list of fusions. }

 

\subsection{\textcolor{Chapter }{AsSet (for orbits of fusions)}}
\logpage{[ 3, 5, 9 ]}\nobreak
\hyperdef{L}{X80C57CB5862EA0E0}{}
{\noindent\textcolor{FuncColor}{$\triangleright$\enspace\texttt{AsSet({\mdseries\slshape fusionorb})\index{AsSet@\texttt{AsSet}!for orbits of fusions}
\label{AsSet:for orbits of fusions}
}\hfill{\scriptsize (method)}}\\


 expands the \textsf{COCO2P}-orbit object \mbox{\texttt{\mdseries\slshape fusionorb}} into a set of fusions. }

 

\subsection{\textcolor{Chapter }{SubOrbitsOfCocoOrbit (for orbits of fusions)}}
\logpage{[ 3, 5, 10 ]}\nobreak
\hyperdef{L}{X7842968983700F79}{}
{\noindent\textcolor{FuncColor}{$\triangleright$\enspace\texttt{SubOrbitsOfCocoOrbit({\mdseries\slshape group, fusion})\index{SubOrbitsOfCocoOrbit@\texttt{SubOrbitsOfCocoOrbit}!for orbits of fusions}
\label{SubOrbitsOfCocoOrbit:for orbits of fusions}
}\hfill{\scriptsize (operation)}}\\


 \mbox{\texttt{\mdseries\slshape group}} is a subgroup of the underlying group of the orbit of fusions \mbox{\texttt{\mdseries\slshape fusionorb}}. The given orbit splits into suborbits under this group. The function returns
a list of these suborbits. }

 

\subsection{\textcolor{Chapter }{SubOrbitsWithInvariantPropertyOfCocoOrbit (for orbits of fusions)}}
\logpage{[ 3, 5, 11 ]}\nobreak
\hyperdef{L}{X7B053C1D7CD14AD8}{}
{\noindent\textcolor{FuncColor}{$\triangleright$\enspace\texttt{SubOrbitsWithInvariantPropertyOfCocoOrbit({\mdseries\slshape group, fusionorb, prop})\index{SubOrbitsWithInvariantPropertyOfCocoOrbit@\texttt{Sub}\-\texttt{Orbits}\-\texttt{With}\-\texttt{Invariant}\-\texttt{Property}\-\texttt{Of}\-\texttt{Coco}\-\texttt{Orbit}!for orbits of fusions}
\label{SubOrbitsWithInvariantPropertyOfCocoOrbit:for orbits of fusions}
}\hfill{\scriptsize (operation)}}\\


 \mbox{\texttt{\mdseries\slshape prop}} is a function that takes a single fusion as argument and returns \texttt{true} or \texttt{false}. It has to be invariant under the partition-wise action of \mbox{\texttt{\mdseries\slshape group}}. Note that the invariance is not checked by the function. 

 This function does the same as 
\begin{Verbatim}[fontsize=\small,frame=single,label=]
  Filtered(SubOrbitsOfCocoOrbit(group,fusionorb), x->prop(Representative(x)));
\end{Verbatim}
 However, the former code is generally much less efficient than calling 
\begin{Verbatim}[fontsize=\small,frame=single,label=]
  SubOrbitsWithInvariantPropertyOfCocoOrbit(group,fusion,prop);
\end{Verbatim}
 }

 }

 }

 
\chapter{\textcolor{Chapter }{Partially ordered sets}}\logpage{[ 4, 0, 0 ]}
\hyperdef{L}{X82C81B747924ED66}{}
{
  
\section{\textcolor{Chapter }{Introduction}}\logpage{[ 4, 1, 0 ]}
\hyperdef{L}{X7DFB63A97E67C0A1}{}
{
  \textsf{COCO2P} implements a data-type for partially ordered sets. The reason is, that for the
posets of interest in \textsf{COCO2P} the test whether two elements are in order-relation is rather expensive, and \textsf{COCO2P} takes care to minimize the necessary tests. The other reason is, that this
approach allows a nice and unified interface to \textsf{XGAP} for all kinds of posets that are introduced in \textsf{COCO2P} (i.e. posets of color graphs, posets of fusion orbits, lattices of fusions,
lattices of closed sets, for now). 

 Like for combinatorial objects, \textsf{COCO2P} internally does not work directly with the elements of a poset, but instead
uses indices into a list of elements (cf. ). Only two functions refer directly
to the elements: \texttt{CocoPosetByFunctions} (\ref{CocoPosetByFunctions}) and \texttt{ElementsOfCocoPoset} (\ref{ElementsOfCocoPoset}). Therefore, in the following, we will identify the index to an element with
the element. }

 
\section{\textcolor{Chapter }{General functions for \textsf{COCO2P}-posets}}\logpage{[ 4, 2, 0 ]}
\hyperdef{L}{X7A784E58852BAB12}{}
{
  

\subsection{\textcolor{Chapter }{CocoPosetByFunctions}}
\logpage{[ 4, 2, 1 ]}\nobreak
\hyperdef{L}{X7F63463D86EF7842}{}
{\noindent\textcolor{FuncColor}{$\triangleright$\enspace\texttt{CocoPosetByFunctions({\mdseries\slshape elements, order, linpreorder})\index{CocoPosetByFunctions@\texttt{CocoPosetByFunctions}}
\label{CocoPosetByFunctions}
}\hfill{\scriptsize (function)}}\\


 This is the main constructor for posets in \textsf{COCO2P}. All other constructors, behind the scenes, use this function. 

 \mbox{\texttt{\mdseries\slshape elements}} is the underlying set of the poset. 

 \mbox{\texttt{\mdseries\slshape order}} is a binary boolean function on \mbox{\texttt{\mdseries\slshape elements}} that returns \texttt{true} on an input pair $(x,y)$ is $x$ is less than or equal $y$ in the poset to be constructed. Otherwise it has to return \texttt{false}. The function \mbox{\texttt{\mdseries\slshape order}} may be algorithmically difficult. 

 \mbox{\texttt{\mdseries\slshape linpreorder}} is a binary boolean function that defines a linear preorder (reflexive,
transitive, total relation) on \mbox{\texttt{\mdseries\slshape elements}}, that extends the partial order relation defined by \mbox{\texttt{\mdseries\slshape order}} such that the strict order of elements is preserved. That is, if $y$ is strictly above $x$ in \mbox{\texttt{\mdseries\slshape order}}, then so it is in \mbox{\texttt{\mdseries\slshape linpreorder}}. 

 The function \mbox{\texttt{\mdseries\slshape linpreorder}} is used to speed up the computations of the successor-relation of the goal
poset. It should be much quicker than \mbox{\texttt{\mdseries\slshape order}} in order to really lead to a speedup. E.g., when computing a poset of sets, \mbox{\texttt{\mdseries\slshape order}} may be the inclusion order, and \mbox{\texttt{\mdseries\slshape linpreorder}} may be the function that compares cardinalities. 

 The function returns a \textsf{COCO2P}-poset object that encodes the poset defined by \mbox{\texttt{\mdseries\slshape order}}. }

 

\subsection{\textcolor{Chapter }{ElementsOfCocoPoset}}
\logpage{[ 4, 2, 2 ]}\nobreak
\hyperdef{L}{X8759C04B85BB9CC0}{}
{\noindent\textcolor{FuncColor}{$\triangleright$\enspace\texttt{ElementsOfCocoPoset({\mdseries\slshape poset})\index{ElementsOfCocoPoset@\texttt{ElementsOfCocoPoset}}
\label{ElementsOfCocoPoset}
}\hfill{\scriptsize (operation)}}\\


 This function returns the list of elements of \mbox{\texttt{\mdseries\slshape poset}}. Indices returned by other operations for posets, will be relative to this
list. }

 

\subsection{\textcolor{Chapter }{Size (for COCO-posets)}}
\logpage{[ 4, 2, 3 ]}\nobreak
\hyperdef{L}{X8429ADC878D1C481}{}
{\noindent\textcolor{FuncColor}{$\triangleright$\enspace\texttt{Size({\mdseries\slshape poset})\index{Size@\texttt{Size}!for COCO-posets}
\label{Size:for COCO-posets}
}\hfill{\scriptsize (method)}}\\


 This function returns the number of elements of \mbox{\texttt{\mdseries\slshape poset}}. }

 

\subsection{\textcolor{Chapter }{SuccessorsInCocoPoset}}
\logpage{[ 4, 2, 4 ]}\nobreak
\hyperdef{L}{X7A4AAC6986FF9BCC}{}
{\noindent\textcolor{FuncColor}{$\triangleright$\enspace\texttt{SuccessorsInCocoPoset({\mdseries\slshape poset, i})\index{SuccessorsInCocoPoset@\texttt{SuccessorsInCocoPoset}}
\label{SuccessorsInCocoPoset}
}\hfill{\scriptsize (operation)}}\\


 This functions returns the successors of \mbox{\texttt{\mdseries\slshape i}} in \mbox{\texttt{\mdseries\slshape poset}}. }

 

\subsection{\textcolor{Chapter }{PredecessorsInCocoPoset}}
\logpage{[ 4, 2, 5 ]}\nobreak
\hyperdef{L}{X79AAF74381042640}{}
{\noindent\textcolor{FuncColor}{$\triangleright$\enspace\texttt{PredecessorsInCocoPoset({\mdseries\slshape poset, i})\index{PredecessorsInCocoPoset@\texttt{PredecessorsInCocoPoset}}
\label{PredecessorsInCocoPoset}
}\hfill{\scriptsize (operation)}}\\


 This functions returns the predecessors of \mbox{\texttt{\mdseries\slshape i}} in \mbox{\texttt{\mdseries\slshape poset}}. }

 

\subsection{\textcolor{Chapter }{IdealInCocoPoset}}
\logpage{[ 4, 2, 6 ]}\nobreak
\hyperdef{L}{X7F4A027C7F395F16}{}
{\noindent\textcolor{FuncColor}{$\triangleright$\enspace\texttt{IdealInCocoPoset({\mdseries\slshape poset, set})\index{IdealInCocoPoset@\texttt{IdealInCocoPoset}}
\label{IdealInCocoPoset}
}\hfill{\scriptsize (operation)}}\\
\noindent\textcolor{FuncColor}{$\triangleright$\enspace\texttt{IdealInCocoPoset({\mdseries\slshape poset, i})\index{IdealInCocoPoset@\texttt{IdealInCocoPoset}!for principal ideals}
\label{IdealInCocoPoset:for principal ideals}
}\hfill{\scriptsize (operation)}}\\


 This function returns the order ideal (a.k.a. downset) generated by \mbox{\texttt{\mdseries\slshape set}} in \mbox{\texttt{\mdseries\slshape poset}}. 

 In the second form, the principal order ideal generated by \mbox{\texttt{\mdseries\slshape i}} in \mbox{\texttt{\mdseries\slshape poset}} is returned. }

 

\subsection{\textcolor{Chapter }{FilterInCocoPoset}}
\logpage{[ 4, 2, 7 ]}\nobreak
\hyperdef{L}{X78F895A9854A84D2}{}
{\noindent\textcolor{FuncColor}{$\triangleright$\enspace\texttt{FilterInCocoPoset({\mdseries\slshape poset, set})\index{FilterInCocoPoset@\texttt{FilterInCocoPoset}}
\label{FilterInCocoPoset}
}\hfill{\scriptsize (operation)}}\\
\noindent\textcolor{FuncColor}{$\triangleright$\enspace\texttt{FilterInCocoPoset({\mdseries\slshape poset, i})\index{FilterInCocoPoset@\texttt{FilterInCocoPoset}!for principal filters}
\label{FilterInCocoPoset:for principal filters}
}\hfill{\scriptsize (operation)}}\\


 This function returns the order filter (a.k.a. upset) generated by \mbox{\texttt{\mdseries\slshape set}} in \mbox{\texttt{\mdseries\slshape poset}}. 

 In the second form, the principal order filter generated by \mbox{\texttt{\mdseries\slshape i}} in \mbox{\texttt{\mdseries\slshape poset}} is returned. }

 

\subsection{\textcolor{Chapter }{MinimalElementsInCocoPoset}}
\logpage{[ 4, 2, 8 ]}\nobreak
\hyperdef{L}{X7B15D8227DCE7F57}{}
{\noindent\textcolor{FuncColor}{$\triangleright$\enspace\texttt{MinimalElementsInCocoPoset({\mdseries\slshape poset, set})\index{MinimalElementsInCocoPoset@\texttt{MinimalElementsInCocoPoset}}
\label{MinimalElementsInCocoPoset}
}\hfill{\scriptsize (operation)}}\\


 This function returns the minimal elements of \mbox{\texttt{\mdseries\slshape set}} in \mbox{\texttt{\mdseries\slshape poset}}. }

 

\subsection{\textcolor{Chapter }{MaximalElementsInCocoPoset}}
\logpage{[ 4, 2, 9 ]}\nobreak
\hyperdef{L}{X852003CF80C95A0B}{}
{\noindent\textcolor{FuncColor}{$\triangleright$\enspace\texttt{MaximalElementsInCocoPoset({\mdseries\slshape poset, set})\index{MaximalElementsInCocoPoset@\texttt{MaximalElementsInCocoPoset}}
\label{MaximalElementsInCocoPoset}
}\hfill{\scriptsize (operation)}}\\


 This function returns the maximal elements of \mbox{\texttt{\mdseries\slshape set}} in \mbox{\texttt{\mdseries\slshape poset}}. }

 

\subsection{\textcolor{Chapter }{InducedCocoPoset}}
\logpage{[ 4, 2, 10 ]}\nobreak
\hyperdef{L}{X815B2EA978621BBF}{}
{\noindent\textcolor{FuncColor}{$\triangleright$\enspace\texttt{InducedCocoPoset({\mdseries\slshape poset, set})\index{InducedCocoPoset@\texttt{InducedCocoPoset}}
\label{InducedCocoPoset}
}\hfill{\scriptsize (function)}}\\


 This function returns the subposet of \mbox{\texttt{\mdseries\slshape poset}} that is induced by \mbox{\texttt{\mdseries\slshape set}} }

 

\subsection{\textcolor{Chapter }{GraphicCocoPoset}}
\logpage{[ 4, 2, 11 ]}\nobreak
\hyperdef{L}{X8347FDA58734E369}{}
{\noindent\textcolor{FuncColor}{$\triangleright$\enspace\texttt{GraphicCocoPoset({\mdseries\slshape poset})\index{GraphicCocoPoset@\texttt{GraphicCocoPoset}}
\label{GraphicCocoPoset}
}\hfill{\scriptsize (operation)}}\\


 This function creates a graphical representation of \mbox{\texttt{\mdseries\slshape poset}} using \textsf{XGAP}. }

 }

 
\section{\textcolor{Chapter }{Posets of color graphs}}\logpage{[ 4, 3, 0 ]}
\hyperdef{L}{X7DA847F581B12DD9}{}
{
  The class of color graphs of order $n$ can be endowed with a preorder relation (i.e. a reflexive, transitive
relation): We say that a color graph \texttt{cgr1} is sub color isomorphic to another color graph \texttt{cgr2} if there is a fusion color graph \texttt{cgr3} of \texttt{cgr2} that is color isomorphic to \texttt{cgr1}. 

 Restricted to a set of mutually non color isomorphic color graphs, the
relation of sub color isomorphism induces a partial order. \textsf{COCO2P} is able to compute this induced order for lists of WL-stable color graphs. 

\subsection{\textcolor{Chapter }{ColorIsomorphicFusions}}
\logpage{[ 4, 3, 1 ]}\nobreak
\hyperdef{L}{X8047079F7D3BC92E}{}
{\noindent\textcolor{FuncColor}{$\triangleright$\enspace\texttt{ColorIsomorphicFusions({\mdseries\slshape cgr1, cgr2})\index{ColorIsomorphicFusions@\texttt{ColorIsomorphicFusions}}
\label{ColorIsomorphicFusions}
}\hfill{\scriptsize (function)}}\\


 This function returns a list of all fusion orbits under the color automorphism
group of \mbox{\texttt{\mdseries\slshape cgr1}} whose representatives induce a color graph that is color isomorphic to \mbox{\texttt{\mdseries\slshape cgr2}}. 

 At the moment this function is implemented only for WL-stable color graphs \mbox{\texttt{\mdseries\slshape cgr1}} and \mbox{\texttt{\mdseries\slshape cgr2}}. }

 

\subsection{\textcolor{Chapter }{SubColorIsomorphismPoset}}
\logpage{[ 4, 3, 2 ]}\nobreak
\hyperdef{L}{X789D5C867BB92EA8}{}
{\noindent\textcolor{FuncColor}{$\triangleright$\enspace\texttt{SubColorIsomorphismPoset({\mdseries\slshape cgrlist})\index{SubColorIsomorphismPoset@\texttt{SubColorIsomorphismPoset}}
\label{SubColorIsomorphismPoset}
}\hfill{\scriptsize (function)}}\\


 \mbox{\texttt{\mdseries\slshape cgrlist}} is a list of WL-stable color graphs all of the same order and no two of them
color isomorphic. The function returns a \textsf{COCO2P}-poset of \mbox{\texttt{\mdseries\slshape cgrlist}} ordered by sub color isomorphism. }

 

\subsection{\textcolor{Chapter }{GraphicCocoPoset (for posets of color graphs)}}
\logpage{[ 4, 3, 3 ]}\nobreak
\hyperdef{L}{X7F8A6835817D9EA1}{}
{\noindent\textcolor{FuncColor}{$\triangleright$\enspace\texttt{GraphicCocoPoset({\mdseries\slshape poset})\index{GraphicCocoPoset@\texttt{GraphicCocoPoset}!for posets of color graphs}
\label{GraphicCocoPoset:for posets of color graphs}
}\hfill{\scriptsize (method)}}\\


 \mbox{\texttt{\mdseries\slshape poset}} is a \textsf{COCO2P}-poset of colored graphs. This function creates a graphical representation of
this poset. The labels of the nodes of the graphical poset correspond to the
indices in the given poset. The context-menu of each node gives additional
information about the node. If for some node it is known whether the
underlying color graph is surian or not, then this is made visible in the
graphical poset. Nodes for which it is not known whether the cgr is Schurian,
are represented by squares. Schurian nodes are represented by circles, and
non-Schurian nodes are represented by diamonds. 

 This function is available only from \textsf{XGAP}. 
\begin{Verbatim}[commandchars=!@|,fontsize=\small,frame=single,label=Example]
  !gapprompt@gap>| !gapinput@lcgr:=AllAssociationSchemes(15);|
  [ AS(15,1), AS(15,2), AS(15,3), AS(15,4), AS(15,5), AS(15,6), AS(15,7),
    AS(15,8), AS(15,9), AS(15,10), AS(15,11), AS(15,12), AS(15,13), AS(15,14),
    AS(15,15), AS(15,16), AS(15,17), AS(15,18), AS(15,19), AS(15,20), AS(15,21),
    AS(15,22), AS(15,23), AS(15,24) ]
  !gapprompt@gap>| !gapinput@Perform(lcgr, IsSchurian);|
  !gapprompt@gap>| !gapinput@pos:=SubColorIsomorphismPoset(lcgr);;|
  !gapprompt@gap>| !gapinput@GraphicCocoPoset(pos);|
  <graphic poset "Iso-poset of color graphs">
  !gapprompt@gap>| !gapinput@|
  	    
\end{Verbatim}
 }

 }

 }

 
\chapter{\textcolor{Chapter }{Color Semirings}}\logpage{[ 5, 0, 0 ]}
\hyperdef{L}{X86F131C387DD3958}{}
{
  
\section{\textcolor{Chapter }{Introduction}}\logpage{[ 5, 1, 0 ]}
\hyperdef{L}{X7DFB63A97E67C0A1}{}
{
  Color semirings are an experimental feature that give an alternate interface
to WL-stable color graphs, in the style of \cite{Zie96} and \cite{Zie05}. 

 In the center stands the observation that the complexes (i.e., subsets of
colors) of WL-stable color graphs can be endowed with a multiplication: Let $\Gamma=(V,C,f)$ be a WL-stable color graph with structure constants tensor $T$, and let $M,N$ be subsets of the color set $C$. Then the product $M \cdot N$ is defined as the set of all colors $k$ such that there exists $i\in M$, and $j\in N$ such that $T(i,j,k)>0$. It is not hard to see that this operation is associative and that the set $I$ of all reflexive colors is a neutral element. Moreover, this product-operation
is distributive over the operation of union of complexes. Thus $(P(C), \cup, \cdot, \emptyset, I)$ forms a so-called semiring (cf. \cite{Gol99}, \cite{wiki:semiring}). 

 The color semiring of $\Gamma$ acts naturally on the powerset $P(V)$ of the vertex set of $\Gamma$ from the left and from the right. Let $C$ be an element of the color semiring, and let $M$ be a set of vertices of $\Gamma$. Then 
\[ C \cdot M := \{ v\in V \mid \exists w \in M : f(v,w) \in C\}, \]
 
\[ M \cdot C := \{ w\in V \mid \exists v \in M : f(v,w) \in C\}. \]
 

 \textsf{GAP} has one operation symbold \texttt{+} for addition-like operations and one operation symbol \texttt{*} for multiplication-like operations. Thus in color semirings, the operation of
union of complexes is denoted by \texttt{+}, and the operation of the product of complexes is denoted by \texttt{*}. 

 Since in \textsf{COCO2P} both, colors and vertices of a color graph are represented by positive
integers, in order to distinguish complexes of colors and subsets of vertices,
one of the two has to get its own type. The elements of color semirings (i.e.,
complexes of colors) all belong to the category \texttt{IsElementOfColorSemiring}. On the other hand, sets of vertices are simple sets of positive integers (no
special category is created for them). In the \textsf{GAP}-output, complexes are denoted like \texttt{{\textless}[ a,b,c ]{\textgreater}}. The conversion of sets of colors to complexes is handled by the function \texttt{AsElementOfColorSemiring} (\ref{AsElementOfColorSemiring}), while the conversion of a complex to a set is done by the function \texttt{AsSet} (\textbf{Reference: AsSet}). 
\begin{Verbatim}[commandchars=!@|,fontsize=\small,frame=single,label=Example]
  !gapprompt@gap>| !gapinput@cgr:=JohnsonScheme(6,3);|
  <color graph of order 20 and rank 4>
  !gapprompt@gap>| !gapinput@T:=StructureConstantsOfColorGraph(cgr);|
  <Tensor of order 4>
  !gapprompt@gap>| !gapinput@sr:=ColorSemiring(cgr);|
  <ColorSemiring>
  !gapprompt@gap>| !gapinput@s2:=AsElementOfColorSemiring(sr,[2]);|
  <[ 2 ]>
  !gapprompt@gap>| !gapinput@s3:=AsElementOfColorSemiring(sr,[3]);|
  <[ 3 ]>
  !gapprompt@gap>| !gapinput@s2*s3;|
  <[ 2, 3, 4 ]>
  !gapprompt@gap>| !gapinput@ComplexProduct(T,[2],[3]);|
  [ 0, 4, 4, 9 ]
  !gapprompt@gap>| !gapinput@1*s2;|
  [ 2, 3, 4, 5, 6, 7, 11, 12, 13 ]
  !gapprompt@gap>| !gapinput@Neighbors(cgr,1,2);|
  [ 2, 3, 4, 5, 6, 7, 11, 12, 13 ]
  !gapprompt@gap>| !gapinput@Neighbors(cgr,1,3);|
  [ 8, 9, 10, 14, 15, 16, 17, 18, 19 ]
  !gapprompt@gap>| !gapinput@1*(s2+s3);|
  [ 2, 3, 4, 5, 6, 7, 8, 9, 10, 11, 12, 13, 14, 15, 16, 17, 18, 19 ]
  	
\end{Verbatim}
 
\begin{Verbatim}[commandchars=!@|,fontsize=\small,frame=single,label=Example]
  !gapprompt@gap>| !gapinput@g:=DihedralGroup(IsPermGroup,10);|
  Group([ (1,2,3,4,5), (2,5)(3,4) ])
  !gapprompt@gap>| !gapinput@cgr:=ColorGraph(g, Combinations([1..5],2), OnSets,true);|
  <color graph of order 10 and rank 12>
  !gapprompt@gap>| !gapinput@ColorMates(cgr);|
  (2,7)(3,10)(6,12)
  !gapprompt@gap>| !gapinput@csr:=ColorSemiring(cgr);|
  <ColorSemiring>
  !gapprompt@gap>| !gapinput@s2:=AsElementOfColorSemiring(csr,[2]);|
  <[ 2 ]>
  !gapprompt@gap>| !gapinput@s3:=AsElementOfColorSemiring(csr,[3]);|
  <[ 3 ]>
  !gapprompt@gap>| !gapinput@1*(s2+s3);|
  [ 2, 3, 6, 7 ]
  !gapprompt@gap>| !gapinput@Neighbors(cgr,1,[2,3]);|
  [ 2, 3, 6, 7 ]
  !gapprompt@gap>| !gapinput@(s2+s3)*[2,3,6,7];|
  [ 1, 4, 5, 8, 10 ]
  	
\end{Verbatim}
 

 Many standard functions of \textsf{GAP} are applicable to color semirings, as a color semiring is just a structure,
that is at the same time an additive magma with zero and a magma with one,
such that multiplication and addition are associative and where the
multiplication is distributive over the addition. 

\subsection{\textcolor{Chapter }{ColorSemiring}}
\logpage{[ 5, 1, 1 ]}\nobreak
\hyperdef{L}{X7B0501A179BCD18C}{}
{\noindent\textcolor{FuncColor}{$\triangleright$\enspace\texttt{ColorSemiring({\mdseries\slshape cgr})\index{ColorSemiring@\texttt{ColorSemiring}}
\label{ColorSemiring}
}\hfill{\scriptsize (function)}}\\


 \mbox{\texttt{\mdseries\slshape cgr}} is a WL-stable color graph. The function returns an object, representing the
color semiring of \mbox{\texttt{\mdseries\slshape cgr}} 
\begin{Verbatim}[commandchars=!@|,fontsize=\small,frame=single,label=Example]
  !gapprompt@gap>| !gapinput@cgr:=JohnsonScheme(6,3);|
  <color graph of order 20 and rank 4>
  !gapprompt@gap>| !gapinput@sr:=ColorSemiring(cgr);|
  <ColorSemiring>
  !gapprompt@gap>| !gapinput@Elements(sr);|
  [ <[  ]>, <[ 1 ]>, <[ 1, 2 ]>, <[ 1, 2, 3 ]>, <[ 1, 2, 3, 4 ]>,
    <[ 1, 2, 4 ]>, <[ 1, 3 ]>, <[ 1, 3, 4 ]>, <[ 1, 4 ]>, <[ 2 ]>, <[ 2, 3 ]>,
    <[ 2, 3, 4 ]>, <[ 2, 4 ]>, <[ 3 ]>, <[ 3, 4 ]>, <[ 4 ]> ]
  !gapprompt@gap>| !gapinput@List(last,AsSet);|
  [ [  ], [ 1 ], [ 1, 2 ], [ 1, 2, 3 ], [ 1, 2, 3, 4 ], [ 1, 2, 4 ], [ 1, 3 ],
    [ 1, 3, 4 ], [ 1, 4 ], [ 2 ], [ 2, 3 ], [ 2, 3, 4 ], [ 2, 4 ], [ 3 ],
    [ 3, 4 ], [ 4 ] ]
  	    
\end{Verbatim}
 }

 

\subsection{\textcolor{Chapter }{GeneratorsOfColorSemiring}}
\logpage{[ 5, 1, 2 ]}\nobreak
\hyperdef{L}{X81E79E24878F491A}{}
{\noindent\textcolor{FuncColor}{$\triangleright$\enspace\texttt{GeneratorsOfColorSemiring({\mdseries\slshape csr})\index{GeneratorsOfColorSemiring@\texttt{GeneratorsOfColorSemiring}}
\label{GeneratorsOfColorSemiring}
}\hfill{\scriptsize (attribute)}}\\


 This function returns a list of additive generators of the color semiring \mbox{\texttt{\mdseries\slshape csr}}. 
\begin{Verbatim}[commandchars=!@|,fontsize=\small,frame=single,label=Example]
  !gapprompt@gap>| !gapinput@cgr:=JohnsonScheme(6,3);|
  <color graph of order 20 and rank 4>
  !gapprompt@gap>| !gapinput@sr:=ColorSemiring(cgr);|
  <ColorSemiring>
  !gapprompt@gap>| !gapinput@gens:=GeneratorsOfColorSemiring(sr);|
  [ <[ 1 ]>, <[ 2 ]>, <[ 3 ]>, <[ 4 ]> ]
  	    
\end{Verbatim}
 }

 

\subsection{\textcolor{Chapter }{AsElementOfColorSemiring}}
\logpage{[ 5, 1, 3 ]}\nobreak
\hyperdef{L}{X80234A84781C367F}{}
{\noindent\textcolor{FuncColor}{$\triangleright$\enspace\texttt{AsElementOfColorSemiring({\mdseries\slshape csr, cset})\index{AsElementOfColorSemiring@\texttt{AsElementOfColorSemiring}}
\label{AsElementOfColorSemiring}
}\hfill{\scriptsize (function)}}\\


 This function takes as input a color semiring \mbox{\texttt{\mdseries\slshape csr}} and a set of colors \mbox{\texttt{\mdseries\slshape cset}}. It returns the element of \mbox{\texttt{\mdseries\slshape csr}} that corresponds to \mbox{\texttt{\mdseries\slshape cset}}. 
\begin{Verbatim}[commandchars=!@|,fontsize=\small,frame=single,label=Example]
  !gapprompt@gap>| !gapinput@cgr:=JohnsonScheme(6,3);|
  <color graph of order 20 and rank 4>
  !gapprompt@gap>| !gapinput@sr:=ColorSemiring(cgr);|
  <ColorSemiring>
  !gapprompt@gap>| !gapinput@s2:=AsElementOfColorSemiring(sr,[2]);|
  <[ 2 ]>
  !gapprompt@gap>| !gapinput@s3:=AsElementOfColorSemiring(sr,[3]);|
  <[ 3 ]>
  !gapprompt@gap>| !gapinput@s2*s3;|
  <[ 2, 3, 4 ]>
  	    
\end{Verbatim}
 }

 }

 }

 \def\bibname{References\logpage{[ "Bib", 0, 0 ]}
\hyperdef{L}{X7A6F98FD85F02BFE}{}
}

\bibliographystyle{alpha}
\bibliography{coco}

\addcontentsline{toc}{chapter}{References}

\def\indexname{Index\logpage{[ "Ind", 0, 0 ]}
\hyperdef{L}{X83A0356F839C696F}{}
}

\cleardoublepage
\phantomsection
\addcontentsline{toc}{chapter}{Index}


\printindex

\newpage
\immediate\write\pagenrlog{["End"], \arabic{page}];}
\immediate\closeout\pagenrlog
\end{document}
